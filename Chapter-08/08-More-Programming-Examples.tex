WFORTH awe—memem 157

More Programming Examples

Contents
51 Numerical integration 157
§§1 The integral of a function 158
552 The fundamental theorem of calculus 159
553 Monte-Carlo method 160
554 Adaptive methods 164
555 Adaptive integration on the real line 165
§§6 Adaptive integration in the Argand plane 179
52 Putting functions to data 181
551 Fast Fourier transform 184
§§2 Gram polynomials 193
553 Simplex algorithm 201
§3 Appendices 204
§§1 Gaussian quadrature 204
552 The trapezoidal rule 205
§§3 Richardson extrapolation 206
§§4 Linear least-squares: Gram polynomials 207
§§5 Non-linear least squares: simplex method 208

n this chapter we apply some of the FORTH tools we have been

developing (complex arithmetic, typed data) to two standard
problems in numerical analysis: numerical integration of a func-
tion over a definite interval; determining the function of a given
form that most closely fits a set of data.

51 Numerlcal Integration

We begin by defining the definite integral of a function f(x). Then
we discuss some methods for (numerically) approximating the
integral. This process is called numerical integration or numeri-
cal quadrature. Finally, we write some FORTH programs based
on the various methods we describe.

omvmim-Mmm.

1 58 Chapter 3 - More Programming Examples Scientific FORTH

“I The Integral ot a function
The definite integral 1?“) dx is the area between the graph of
the function and the x-axis as shown below in Fig. 8-1:

 

 

 

firm dx

 

 

 

 

 

 

 

Flg. 8-1 The Integral of a function Is the area under me curve.

We estimate the integral by breaking up the area into narrow
rectangles of width w that approximate the height of the curve at
that point and then adding the areas of the rectanglesl. For
rectangles of non-zero width the method gives an approximation.
If we calculate with rectangles that consistently protrude above
the curve (assume for simplicity the curve lies above the x-axis),
and with rectangles that consistently lie below the curve, we
capture the exact area between two approximations. We say that
we have bounded the integral above and below. In mathematical

language.

 

l. Hareaanglefleshdwmehorizoutdamiuueaiscomidcredtobeupdn.

(b-‘yw
w 2 min [fla+nw),fla+nw+w)]

ad)

5 fitted: (1)

(b-a)/w
s w 2 max [fla+nw),fla+nw+w)]
u-O
It' is easy to see that each rectangle' m the upper bound IS about
W U" (x) I too high2 on the average, hence overestimates the area

by about 5w 2If" (x)|. There are (b—a)/w such rectangles, so if

If '(x) I remains finite over the interval [a, b] the total discrepancy
will be smaller than

—W(b-a) max If' (X)|-
ansb

Similarly, the lower bound will be low by about the same amount.
This means that if we halve w (by taking twice as many points),
the accuracy of the approximation will double. The mathematical

definition of f:f(x) dr is the number we get by taking the limit as

the width w of the rectangles becomes arbitrarily small. We know
that such a limit exists because the actual area has been captured
between lower and upper bounds that shrink together as we take
more points.

552 mmmmolmlcum

Suppose we think of It‘/Ix) dz as a function —eall it F(b) — of

the upper limit, b. What would happen if we compared the
area F(b) with the area F(b + Ab): We see that the difference
between the two is (for small Ab)

AFC?) = F(b+Ab) -F(b) “NON? +0((Ab)2) (2)

 

2.

f'(x)isthealopedthelinetangenttothecmveatthepointx.lt'ucalledtheflnth1vaflveof

fix).

so that

- 1 +Ab
F'b = _ [b _[b
() hm0 Ab( . dz .flx)dx) (3)

Equation 3 is a fancy way to say that integration and differentia-
tion are inverse operations in the same sense as multiplication
and division, or addition and subtraction.

This fact lets us calculate a definite integral using the differential ‘
equation routine developed in Chapter 6. We can express the
problem in the following form: ‘

Solve the differential equation

(11’
I =f(x) (4)
fromx = a tox = b, subject to the initial condition

 

F(a) = 0.
The desired integral is F (b).

The chief disadvantage of using a differential equation solver to
evaluate a definite integral is that it gives us no error criterion
We would have to solve the problem at least twice, with two
. . . . . 3
different step Slzes, to be sure the result 15 sufficxently precise .

§§3 Monte-Carlo method
The area under f(x) is exactly equal to the average heightfof
f(x) on the interval [a, b], times the length, b—a, of the interval .
How can we estimate]? One method is to sample f(x) at random.

 

This is not strictly correct: one could use a differential equation solver of the “predictor/come-
tor" variety, with variable step-size, to integrate Eq. 4. See, e.g., Press, er al., Numerical Recipe:
(Cambridge University Press, Cambridge, 1986), pp. 102 if.

That is, this statement deflnesf.

Ml—MPWEW 101

choosing N points in lab] with a random number generator.
Then

1 N
7" ‘fi Iglflxn) (5)
and
flax) dx - (b — a)? (6)

This random-sampling method is called the Monte-Carlo method
(because of the element of chance).

“3-1 UncertaintyottheMonte-Canomethod

The statistical notion of variance lets us estimate the accuracy of
the Monte-Carlo method: The variance in f(x) is

W) =If:p(f)(f-r)2df
N (7)
=5): (run—IV

n-l

(here p(f )df is the probability of measuring a value off between
f and f + df).

Statistical theory says the variance in estimating 7 by random
sampling is

Vera) = fivw) (8)

Le, the more points we take, the better estimate offwe obtain.
Hence the uncertainty in the integral will be of order

A(Iff(x)dr) Jaw: (9)

and is therefore guaranteed to decrease as W'

162

CMpterB—MoreProgrnnmflngExnmplea SclentlflcFORTH

It is easy to see that the Monte-Carlo method converges slowly.

. 1
Smce the error decreases only as W , whereas even so crude a

rule as adding up rectangles (as in §1§§1) has an error term that
decreases as UN, what is Monte-Carlo good for?

Monte-Carlo methods come into their own for multidimensional
integrals, where they are much faster than multiple one-dimen-
sional integration subroutines based on deterministic rules.

§§3—2 A slmple Monte-Cane program

Following the function protocol and naming convention devel-
oped in Ch. 6 §1§§3.2, we invoke the integration routine via

USE( Ename % L.lim 96 U.lim % err )MONTE

We pass )MONTE the name Ename of the function f(x), the
limits of integration, and the absolute precision of the answer.
The answer should be left on the ifstack. L. Ilm, U. Ilm and err
stand for explicit floating point numbers that are placed on the
87stack by 965 Th.e word 96 appears explicitly because 1n a larger
program — of which )MONTE could be but a portion— we might
want to specify the parameters as numbers already on the 87stack.
Since this is intended to be an illustrative program we keep the
fstack simple by defining SCALARs to put the limits and
precision into.

3REAL'4 SCALARS A B-A E

The word INITIALIZE will be responsible for storing these num-
bers.

The program uses one of the pseudo-random number generators
(prng’s) from Ch. 3 55. We need a word to transform pm’s
—uniformly distributed on the interval (0,1)— to prn's on the
interval (A, B):

 

‘I‘hewordfi‘ pusheswhatfollomintheinputstreamontotheflstack,assumingiteanbeinter-
pretedasalloatingpointnumber.

OJdlanVNoblefm-Mmmorved.

1...»...

 

Immm MLMPWEW 163

:NEW.X RANDOM B-A G@ F‘ A G@ F+;

The program is described by the simple flow diagram of Fig. 8-2
below:

 

INITIALIZE

db

BEGIN
(B-A)'sigma > E 7
WHILE
New.x f(x)
N=N+1
T. Var(f)
REPEAT
I = (B-A)*<f>

 

 

 

 

 

Fig. 8-2 FIow diagram of Monte Carlo Integration

From the flow diagram we see we have to recompute f and
Var(f' ) at each step. From Eq. 5 we see that

In“ =7»: + fl—xrpfl Iii—N

and

_ _ - V
VaIN+1 = VarN + UN” IN)UI;;I+ 171s“) arN

Writing the program is almost automatic:

:USE( [COMPILE] ' CFA UTERAL ; IMMEDIATE
3REAL‘4 SCALARS AV.F dd.Av.F VarF

:DoAvernge (n--n+1B7:f--t)
Av.F G@ ddAvF GI \seveoldAv
FDUP 1+ (--n+187:--fl)
oldAvF G@ (87:--ffddAv.F)

FUNDER F-
DUP s->F Fl F+ (87:--fAv.F)
MFGI ; \pusway \corl'dbdow

OJdIanVNmfm-Alrldnareurved.

Chapter 8 - More Programming Exnrnpleo

:Do.Verhnce (n--n 37:1")
FDUP old.Av.F G@ (87:ffold.Av)
FUNOER F— FSWAP
Av.F G@ F- F'

(37: [l-Oid.AV]‘[l~AV] )
VarF G@ FUNDER F-

DUP S->F Fl F+(87:--Var’)
VarF G!
:INITIAIJZE (:adr-- 87zabe --)
IS adrf
E G!
FOVEFI F- B-A G! A G!
FINIT
F=0 VarF G!
F=0 Av.F G!
F=0 oId.Av.F G!

o 5 COO \exerclseStImes

 

I

Sclendfic FORTH

NEW)! adrf EXECUTE
DoJNernge Do.Varhnce
LOOP ;

:NotConverged? VarF G@ FSORT

B-AG@F"EG@F>;

2 DEBUG DUP

10 M00
0= IF CR DUP.
Av.F G@ F.
VarF G@ F. THEN ;

\every1oaepe

:)MONTE

INITIAUZE

BEGIN DEBUG NotConverged?

WHILE NEW.X adr.f EXECUTE
Do.Average Dome

REPEAT

DROP Av.F G@ B-A G@ F';

§§4 Adaptlve methods

The word DEBUG is included to produce useful output every 10
points as an aid to testing. The final version of the program need
not include DEBUG, of course. Also it would presumably be ‘
prudent to BEHEAD all the internal variables.

The chief virtue of the program we have just written is that it is
easily generalized to an arbitrary number of dimensions. The
generalization is left as an exercise.

Obviously, to minimize the execution time of an integration
subroutine requires that we minimize the number of times the 3
function f(x) has to be evaluated. There are two aspects to this:

0 First, we must evaluate f(x) only once at each point x in the

interval. .

TI

0 Second, we evaluate f(x) more densely where it varies rapidly
than where it varies slowly. Algorithms that can do this are
called adaptive.

To apply adaptive methods to Monte Carlo integration, we need:

mafia

 

an algorithm that biases the sampling method so more points are :5

W
a.

?

eJuunVNounm-Arngmrmmd. f‘

WM?"

“5

Ml-MWW 1“

chosen where the function varies rapidly. Techniques for doing
this are known generically as stratified sampling6.111e difficulty
of automating stratified sampling for general functions puts adap-
tive Monte Carlo techniques beyond the scope of this book.

However, adaptive methods can be applied quite easily to deter-
ministic quadrature formulae such as the trapezoidal rule or
Simpson’s rule. Adaptive quadrature is both interesting in its own
right and illustrates a new class of programming techniques, so
we pursue it in some detail.

Adaptive Integration on the real line

We are now going to write an adaptive program to integrate an
arbitrary function flx), specified at run- -time, over an
arbitrary interval of the x- axis, with an absolute precision
specified in advance. We write the integral as a function of several
arguments, once again to be invoked following Ch. 6 §1.3.2:

USE( Ename % L.lim % U.lim 96 err )INTEGRAL

Now, how do we ensure that the routine takes a lot of points when
the function f(x) is rapidly varying, but few when f(x) is smooth?
The simplest method uses recursion7.

“5-1 Dlgresslon on recursive algorithms

We have so far not discussed recursion, wherein a program calls
itself directly or indirectly (by calling a second routine that
then calls the first).

Since there is no way to know a priori how many times a program
will call itself, memory allocation for the arguments must be
dynamic. That is, a recursive routine places its arguments on a

 

6.
7.

.I.M. Hammersley and DC. Hanscomb, Monte Carlo Methods (Methuen, london, 1964).
See, ¢.g., R. SedgewichAWn: (Addison-Wesley Publishing Company, Reading, MA,

19113), p. as.

166

Chapter 11 — Mora Programing Examples Scientific FORTH

stack so each invocation of the program can find them. This is the
method employed in recursive compiled languages such as Pascal,
C or modern BASIC. Recursion is of course natural in FORTH
since stacks are intrinsic to the language.

We illustrate with the problem of findin the greatest common
divisor (gcd) of two integers. Euclid devised a rapid algo-

rithm for finding the gcd9 which can be expressed symbolically
as

_ u, v=0
gcd 0‘” ‘ {gcd (v,u mod v) else (10)

That is, the problem of finding the gcd of u and v can be replaced
by the problem of finding the gcd of two much smaller numbers.
A 1:0an word that does this 151°

:GCD (uv--9cd)
?DUP 0 > \ stopping criterion
IF UNDER MOD RECURSE THEN ;

Here is a sample of GOD in action, using TRACE11 to exhibit the
rstack (in hex) and stack (in decimal):

 

new

11.

An ancient Greek mathematician known for one or two other things!

Sec, e.g. Sedgewick, op. cit, p. 11.

Most FORTHs do not permit a word to call itself by name; the reason is that when the compiler
tries to compile the self-reference, the definition has not yet been completed and so cannot be
looked up in the dictionary. Instead, we use RECURSE to stand for the name of the self-calling
word. See Note 14 below.

TRACE is specific to HS/FORTH, but most dialects will support a similar operation. SSTRACE
is a modification that sinde-stcps through a program.

MM"!

MI—MWW

marmosets:

187

m and:
:GCO 784 48
M 784 48 48
0- 784 48 0
W<8>0 784 48
“DER 48 784 48
MOD 48 16
:00) 48 16
M 4878 48 18 18
0- 4878 48 18 0
m<8>0 4878 48 18
UNDER 4878 18 48 18
MG) 4878 18 0
.600 4878 18 0
DUP 4878 4878 18 0 0
= 4878 4878 18 0 6535
OBRANCH<B>-1 4878 4876 16 0
DROP 4878 4878 18
EXIT 4876 16
EMT 18

Note how GCD successively calls itself, placing the same address
(displayed in hexadecimal notation) on the rstack, until the stop-
ping criterion is satisfied.

Recursion can get into difficulties by exhausting the stack or
rstack. Since the stack in 600 never contains more than three
numbers, only the rstack must be worried about in this example.

ecursive programming possesses an undeserved reputation

or slog execution, compared with nonrecursive equivalent
programs . Compiled languages that permit recursion —e.g.,
BASIC, C, Pascal— generally waste time passing arguments to
subroutines, Le. recursive routines in these languages are slowed
by parasitic ailing overhead. FORTH does not suffer from this
speed penalty, since it uses the stack directly.

 

12. Pmenmphhhofladaimedthnnmovingmeuaionahnoualwaysproduwsafmeulgo
rithm.See,e.g.Sedgewick,ap.ciL,p.12.

ammonium-Arm“.

ChapterB—MoraProgmmming Exam Scientific FORTH

Nevertheless, not all algorithms should be formulated recursive-
ly. A disastrous example is the Fibonacci sequence

F0 = 0, F1: 1, F“ = Fn—l '1' Fn-2 (11)
expressed recursively in FORTH as

:FIB (:n--F[n])
DUP 0> NOT
IF DROP o EXIT THEN

IF DROP 1 EXIT THEN \n > 1

1- DUP 1- (-- n-1 n-2)
RECURSE SWAP (- - F[n-2] n-1 )
RECURSE + ;

This program is vastly slower than the nonrecursive version
below, that uses an explicit DO loop:

:FIB (:n--F[n])
0 1 HOT (:01 n)
DUP O> NOT
IF DDROP EXIT THEN
DUP 1 =

IF DROP PLUCK EXIT THEN
1 DO UNDER + LOOP PLUCK ;

Why was recursion so bad for Fibonacci numbers? Suppose the
running time for F n is T,, ; then we have

T, ~ T.,_1 + T,,_2 + r (12)

where r is the integer addition time. The solution of Eq. 12 is

n
r, = «Dz—‘5) — 1] (13)

That is, the execution time increases exponentially with the size
of the problem. The reason for this is simple: recursion managed
to replace the original problem by two of nearly the same size, Le.
recursion nearly doubled the work at each step!

The preceding analysis of why recursion was bad suggests how ,
recursion can be helpful: we should apply it whenever a given :

OJUIInVNobletm-Nirigflsrsssrwd.

 

ms-MWW 189

problem can be replaced by -say- two problems of half the
original size, that can be recombined in n or fewer operations. An
example is mergesort, where we divide the list to be sorted into
two roughly equal lists, sort each and then merge them:
subroutine sort(|ist[0,n])

partition' (list, 11811, 11312)

sortaistl)

sort(list2)

mergeaisn, list2, list)

In such cases the running time is
Tn'Tv2+Tv2+fl=Zruz+n (14)
for which the solution is
Tn~nlog1(n) (15)

(In fact, the running time for mergeson is comparable with the
fastest sorting algorithms.) Algorithms that subdivide problems
in this way are said to be of divide and conquer type.

Adaptive integration can be expressed as a divide and conquer

algorithm, hence recursion can simplify the program. In pseu-
docode (actually QuicltBasicO ) we have the program shown
below.

function simpson(f, a, b)

c = (a + b)/2

simpson = (1(a) + 1(b) + 4*f(c)) " (b - a) /6
end function

function integral(f, a, b. error)
e = (a + b)/2
oid.int = simpson(i, a, b)
new.int = simpson(f, a, c) + simpson(1, c, b)
it abs(oid.int - new.int) < error than
integral = (16*new.int - oid.int) /15
else
integral = integral(f, a, c. error/2) +
integral(f, c, b. error/2)
end if

endtunction

emvmmpumm.

170

Chapter 8 — More Programming Examples Scientific FORTH

Clearly, there is no obligation to use Simpson's rule on the sub-
intervals: any favorite algorithm will do.

To translate the above into FORTH, we decompose into smaller
parts. The name of the function representing the integrand (ac-
tually its execution address or cfa) is placed on the stack by USE( ,
as in Ch. 8 §1.3.2 above. Thence it can be saved in a local variable
—either on the rstack or in a VAR or VARIABLE that can be
BEHEADed — so the corresponding phrases

R@ EXECUTE \rstack
name EXECUTE \VAFI
name EXECUTE@ \VARIABLE

evaluate the integrand. Clearly the limits and error (or tolerance)
must be placed on a stack of some sort, so the function can call
itself. One simple possibility is to put the arguments on the
87stack itself. (Of course we then need a software fstack manager
to extend the limited 87stack into memory, as discussed in
Ch. 4 §7.) Alternatively, we could use the intelligent fstack
(ifstack) discussed in Ch. 5 §2.5. We thus imagine the fstack to be
as deep as necessary.

The program then takes the form13 shown on p. 171 below.

Note that in going from the pseudocode to FORTH we replaced
)INTEGRAL by RECURSE inside the word )INTEGRAL The
need for this arises from an ideosyncrasy of FORTH: normally
words do not refer to themselves, hence a word being defined is
hidden from the dictionary search mechanism (compiler) until
the final ; is reached. The word RECURSE unhides the current
name, and compiles its cfa in the proper spot“.

 

13.

14.

For generality we do not specify the integration rule for sub-intervals, but factor it into its own
word. If we want to change the rule, we then need redefine but one component (actually two,
since the Richardson extrapolation —see Appendix 8.C— needs to be changed also).

We may define RECURSE (in reverse order) as
2 RECURSE 7COMP LAST-CPA , ;
: 7COMP STATE @ 0 1: ABORT“ Compile only!“ ;
: LAST-CPA LATEST PFA CFA : IMMEDIATE
\ These defintions are appropriate for HS/FORTH
Note that RECURSE is called MYSELF in some dialects.

 

MM"!

:USE( [Wtfl'GAUTEML;
MATE

1100 (:da--da) DU’EXEQJTE;

1W (1*- Ibu')
\taumpuoudm
mu» m- F21 (er-411mm)
F-RO‘I’ rot) FSWAP rot) F+ F‘ ;
:)Rich-dam \R-odrmJortrapnle
as->1= Fl F+ ; (av: r1--1")

DVAHABLE ERR \ place to store an
CREATE OLD.I 10 ALLOT
\ [lace to storeilab]

 

MLMPWEW 171

:MNTEGHAL (run 87: lbw—J)

EM P321
W W
OLDJ W
XDUP F+ F2I(87: --abc-[a+b1/2)
FUNDEFI FSWAP (87: --ac c b)
XDUP )htegiwr --ac c b 11)
HP HP (87: --ac c b I1ac)
)Intogral F+ (s7: --ac c b 11+12)
FDUP OLDI W F-
FDUP FABS EM 332@ F<
IF )Rlctnrtiaon
FPLUCK FPLUCK FPUJCK FPUJCK
ELSE FDROP FDROP (87: --ac c b)
ERR 832@ F2!
F-ROT FZP (87: --ac err/‘2 cben/Z)
RECUFISE (87:--acerr/2 I[c,b])
F3R F3FI F311 RECURSE F+
THEN DROP ;

(87: -- lb 0)

“5-2 Disadvantages of recursion in adaptive Integration
The main advantage of the recursive adaptive integration algo-
rithm is its ease of programming. As we shall see, the recursive
program is much shorter than the non-recursive one. For any
reasonable integrand, the fstack (or ifstack) depth grows only as
the square of the logarithm of the finest subdivision, hence never

gets too large.

However, recursion has several disadvantages when applied to

numerical quadrature:

e The recursive program evaluates the function more times than

necessary.

0 It would be hard to nest the function )INTEGRAL for multi-

dimensional integrals.

Several solutions to these problems suggest themselves:

0 The best, as we shall see, is to eliminate recursion from the

algorithm.

a We can reduce the number of function evaluations with a more
precise quadrature formula on the sub-intervals.

OWVNoUolm-me.

172

Chapter 11 — More Programming Examples Scientific FORTH

e We can use “0 n" formulas like Gauss-Legendre, that omit
the endpoints see Appendix 8.1).

§§s-3 Adaptive Integration without recursion

he chief reason to write a non-recursive program is to avoid

any repeated evaluation of the integrand. That is, the optimum
is not only the smallest number of points it, in (A, B] consistent
with the desired precision, but to evaluate f(x) once only at each
x... This will be worthwhile when the integrand f(x) is costly to
evaluate.

To minimize evaluations of f(x), we shall have to save values
f(x,,) that can be re-uscd as we subdivide the intervals.

The best place to store the f(x,,)’s is some kind of stack or array.
Moreover, to make sure that a value of f(x) computed at one
mesh size is usable at all smaller meshes, we must subdivide into
two equal sub-intervals; and the points 1:“ must be equally spaced
and include the end-points. Gaussian quadrature is thus out of
the question since it invariably (because of the non-uniform
spacings of the points) demands that previously computed f(x,,)’s
are thrown away because they cannot be re-used.

The simplest quadrature formula that satisfies these criteria is the
trapezoidal rule (see Appendix 8.2). This is the formula used in
the following program.

To clarify what we are going to do, let us visualize the interval of
integration, and mark the mesh points (where we evaluate flx)

mm+z

5__m

XIMMWWEEM
xr1 fo
X1 Ty lo 8

summer" MI—MWW 173

We now save (temporarily) lo and divide the interval in two,
computing [’0 and I, on the halves, as shown. This will be one

fundamental operation in the algorithm.
Step 2: N - N + 1 - 2

 

 

 

A l'o I, B
I l l
I I I
X0 Xi X:

x ttxlintaerals. Error
"o In

X. 7' 1'0 8’2
x2 Ia I1 8/2

We next compare I’o +11 with It, The results can be expressed as
a branch in a flow diagram, shown below.

 

 

 

 

 

Yes I No
Accumulate: Subdivide right-most
I“ a I”, ‘1" I’N-1 +1" interval
N a: "-2 N = N + 1
Move everything down

 

 

 

 

 

 

I I

 

 

 

Fig. 8-3 SUBDMDE branch In adaptive Integration

If the two integrals disagree, we subdivide again, as in Step 3 and
Step 4 below:

174

Step3: N=N+1=3

W" 3 - MON Programming Examples Scientific FORTH It"

 

 

 

 

 

 

 

 

 

 

 

A In I" I; —.B
I I I J
I I I j
X0 X1 X2 X.
)1 Minimal: Error
x0 10
x1 f1 I0 8/2
X2 f2 I11 5/4
x3 1, 12 5/4
Step 4: N=N+1=4
A 1,, I, 1', 13—.B
I -I . . I l I
I ..I , T T I
X0 X1 X2 X3 X4 .3
x tixllnteorals. Error
x0 10
X1 T1 '0 En
X2 f2 '1 8/4 .
his 1.2-- 5/8..
X4 T4} ' I3 5/8- .
Now suppose the last two sub-integrals (13 + 1’2) in step 4 agreed
with their predecessor (12); we then accumulate the part com-
puted so far, and begin again with the (leftward) remainder of the
interval, as in Step 5:
Step 5: N=N-2=2 I=l+(l'2+I3')'+ (l'2+l3-I2)/3
A lo I, B
l ' I I I
I I I DONE
X0 X1 X2 I
x winteotala Error ‘
x0 10
X1 f1 Io 5/2

The flow diagram of the algorithm now looks like Fig. 84 below: i _

OJtlthNobietm—Mngilsreserved.

 

mum owe-mammary. 175

 

I Initialize

 

® BEGIN
Y“ N 0>
WHILE
01d.I = IN-l
SUBDIVIDE Converged?
Yes
IF Richardson_interpoiate
No store last 2 subintegrais

N=N-2

 

 

 

 

 

 

Fig. 84 Non—recursive adaptive quadrature

and the resulting FORTH program is 15:

I15. WeusethegeneralizedarraysofCh.S§3.4;A,BandEarefp I'sonthefstuLTYPEisthe
dill—Wolf(x)lndmm

 

i

omvmrm-umm.

176 Channfl-MorePrognmthnm

\ COPYRIGHT 1991 JUUAN v. NOBLE
TASK IN1EGRAL
FIND CP@L o= ?( FLOAD COMPLEX)
\ define data-type tokens 11 not already
FIND REAL-4 o = 7(((

o CONSTANT REAL-4

1 CONSTANT REAL'B

2 CONSTANT COMPLEX

a CONSTANT DCOMPLEX )))

FIND 1ARRAY 0= ?(FLOAD MATRIXHSF )
\function usage
:USE( [OOMPILE] ' CFA ; IMMEDIATE

\ BEHEADing starts here
0 VAR N

:inc.N N1+ ISN;
:dec.N N2- ISN;

OVAR type

\ define “stack"
20 LONG REAL-a 1ARRAY X{
20 LONG REALM 1ARRAY E{
20 LONG DCOMPLEX 1ARRAY F{
20 LONG DCOMPLEX 1ARRAY 1{

2 DCOMPIEXSCALARS old.| flmIJ
:)imegral (n--)\trapezoidal n19
X{ OVER } G@L
X{ OVER 1- } G@L
F- F2/
F{ OVER } G@L
F{ OVER 1-} G@L
typez AND
IF X+ FROT X‘F
ELSE F+ F' THEN
I{ SWAP 1- } GlL ;
0VARI.mme
: f(x) tname EXECUTE ;

 

samronm
:INI‘HNJZE

IStype\m1ype
typoF{l
type|{ l\oettypesforfma|on
type 'oldJ!
typo 'flrfl.” \andflewaKs)
type 1 AND X{!

\Satypeforhdep.var.
E{0} G!L \storeenot
x{1}G!L \storeB
X{O} G!L \sloraA
IS name \!claoff(x)

X{O}G@L 1(x) F{O} G!L
X{1}G@L f(x) P{1}G1L

1ISN

N)integrd

typezAND IF F=0 THEN
F=0 finall G!L

FINrT ;

:E/2 E{N1-} G@L F2! E{N1-} G!L:

:}move.down (adrn--)

} #BYI'ES >R (usegofi)
DDUP R@ +
(--s.seg S.ofl d.sag doll)
R> CMOVEL ;
: MOVEDOWN

E{ N 1— }move.down
X{ N }move.down
F{ N }mova.dcmn ;

:new.X (87:--x')
X{N}G@L X{N1-}G@L
F+ F2! FDUP X{N} G!L ;

\Conl’d.

MM"!

\NTEGflALm'd
\mm
:0F.1> IFFSWAP E.Tl-ENE.:
:FO. DLP>R G@L R> 0F. ;
:.8TAO<S CR ."N"
OCTAB ."X'
19cm .~Ro{Poor
31CTAB."Im{F(X)]"
«cm ."Rem"
57CTAB."|m[I]"
71CTAB."E"
N2+ 000 OR I.
3CTAB X{I}F@.
tecTAB F{I} F@.
42CTAB |{l} F@.
escrAa E{I} F@.
LOOP
CR SSPACES."dd.I =" dd.l F@.

SSPACES ."MJ =“WJ F@. CR ;

USE: <MBUG> NEXT.STACKS :GASE
DVAR (DEBUG)
:W

:DasUG (DEBUG) <DEBUG>;

:SUBDMDE
N19> ABORT‘Toonanyubdeom!"
EIZ MOVEDOWN
I{N1-} DROP ddl #BYTES CMOVEL
nan for) F{N}GIL
N)htemi N1+)negtu;

Ml-MPWW

1IS(DBUG)5#PLACESI :
:EBUG-CFF OIS (DEBUG) 7#PLACES! ;

 

177

:CCNVERGED? (C7:--l[N]H'[N-1H[N—1]z--n
'(MGOLHM-mot unmet
type 2 Am
IF CP- CP+ CPDUP CPAas
ELSE F- F+ FDUP FABS
THEN
E{Nt-)G@L Fr F<;

CASE: 9'! CP'F F' :CASE
4S->F 38->F Fl FWNSTANT Fad/3

:INTERPOMTE (G7: I[N] +I'[N—1]4[N-1]- -)
F-4/3 type 2/ 9'!
old.l G@L N.IG@L
typeZAND
IF CP+ CP+
ELSE F+ F+
W.IG!L ;
\BEHEADhgendshera

THEN

:)INI'EGRAL (37: A B ERR - - I[A.B])
INI'HNJZE
BEGIN N0>
WHILE SUBDMDE DEBUG
CONVERGED? th
IF INTERPOLATE dec.N
ELSE typezAND IF FDROP
THEN FDROP
THEN
REPEAT WA G@L :
BEHEAD" N INTERPOLATE \oflw
\USE(F.name%A%B%E type)lNTEGRAL

The nonrecursive program obviously requires much more code
than the recursive version. This is the chief disadvantage of a

nonrecursive method

 

11‘. The memory usage is shout the same: the recursive method pushes limits, etc. onto the (suck.

emvNounm—Almw.

“5-4 Example of )INTEGRAL IN USE

Chapter 3 - More Programming Exampteo Scientific FOR? ,

   
  

he debugging code (“DEBUG-ON") lets us track the executio
of the program by exhibiting the simulated stacks. Here is a
2

example, fdr t/f:
1
USE( FSQRT 96 1. 96 2. 96 1.5-3 REAL'4 )INTEGRAL E

0 LMH'D 1.0000E+oo 5.5618501 50113504
1 1.5m-H'D 12247E+w 8.5973501 5.0mm
2 zoomsmo 1.41425+m 1.433501 125M504
old.l=1.m71E+w finalJ = omooE+oo

0 1.ooooE+oo tm-H'D 5.5618E01 50000an
1 1.5m+m 12247504!) 3.1845E01 250001504
2 1.7503544!) 1.32%E+CD 3.4213501 ZSGDE04
3 2.ooooe+oo 1.4142£+m 1.7m01 12513504
oldJ = 6.$73501 finalJ = o.omoE+oo

O LOWEH'D LMHD 5.5618501 sonooeoa
1 1.5GDE+CIJ 1.2247E+CD 3.1845501 25000604
2 1.7EKDE+m 1323500 1.6825501 125W504
3 1.8750E+(XJ 1.3ME+(D 1.7m501 1mm

4 zooooE+oo 1.414ZE+m o.ooooE+oo o.ooooE+oo

old] = 3.4213501 finall = oamE+oo

O IQIDEH'D 1QIDE+CD 5.5618501 sooooem
1 1.5m+(0 12247E+m 1.5&1501 1M“
2 1.650900 12747E+m 1.6M01 12500604
3 1.75mE+CD 1.3235H'D 1.7ME01 1.2511504
oldJ = 3.1845E0‘I final! = 3.42%E01

 

I E N X F I E

O 1.m+m tm+m 2.6475501 251111504
1 12500E+oo 1.11&£+m 2.9284501 2mm
2 1.5GDE+W 12247E+m 1.625501 1.251504
old.| = 5.5818501 fin-LI = 8.EB7E01

O tm+m 1.m+m 2.8475E01 M414
1 1.25an+oD 1.11wE+w 1.4316501 tzsmeoa
2 Lam-HI) 1.1723E+oo 1.4mm 1mm
3 1.m5+m 12247E+m 1.7m01 1mm
old.| = 2.9234501 finllJ = 6.m01

o1.ooaoE+ao 1.0000E+0012579E01 1251504
1 1.1250544!) tmE-HD 1.3616501 1.231204
2 12500E+oo 1.11wE+CD 1.4mm 12mm
oldJ = 2.6475501 finalJ = 9.5ME01
1.21&E+m 0k

\

 

dx1/'=§ (23/2—1)

dSN

 

 

 

 

Notice that, although t/f is perfectly finite atx = 0, its first derivav
tive is not. This is not a problem in the above case, because the:
lower limit is 1.0.

It is an instructive exercise to run the above example with the'
limits (0.0, 1.0). The adaptive routine spends many iterations:
approachingx = 0 (25 in the range [0,, 0.0625] vs. 25 in the ranger:
[0.0625, 1.0] ). This is a concrete example of how an adaptive-1
routine will unerringly locate the (integrable) singularities of at
function by spending lots of time near them. The best answer to:
this problem is to separate out the bad parts of a function by hand...
if possible, and integrate them by some other algorithm that take
the singularities into account. By the same token, one shoul
always integrate up to, but not through, a discontinuity in f(x).

WFORTH MI-mwem 179

"6 mmmmmm

We often want to evaluate the complex integral

1 - +1 «21.1: (16)

where I‘ is a contour (simple, closed, piecewise tinuous curve)
in the complex z-plane, and 1(2) is an analytic1 function of z.

The easiest way to evaluate 16 is to parameterize 2 as a function
of a real variable t ; as 1 runs from A to B, z(t) traces out the
contour. For example, the parameten'zation

z(t) = 20 + R cos(t) + iR sin(t) , 05152: (17)

traces out a (closed) circle of radius R centered at 2 =20

We assume that the derivative 2(t) E E can be defined; then the

dt
integral 16 can be re-written as one over a real interval, with a
complex integrand:
I = Ii 2(1)f(z(1)) d: (18)

Now our previously defined adaptive function )INTEGRAL can
be applied directly, with Emma the name of a complex funcan

:0) = 2(t)f(z(r)). (19)
of the real variable I.

Here is an example of complex integration: we integrate the

function flz) = el/z around the unit circle in the counter-clock-
wise (positive) direction.

 

7. “Analytic"meanstheordinaryderivativedf(z)/dzeidsts.Consultanygoodtenonthetheoryof
Wdampkxvariable.

OJUUIVNOUOIM-Almm.

160

The calculus of residues (Cauchy's theorem) gives

é 4n” = 2771'
Iz|=1

We parameterize the unit circle as z(t)- — cos(27rt) + i sin(2.m
hence z(t) = 27tiz(t), and we might as well evaluate

I; dtz(t) 21/2“) E 1.

For reasons of space, we exhibit only the first and last few iterationst

ChapterO-MoraProgrammIngEnmpIoa

semmmeronn

(20)

(21)

 

I

FIND FSINCOS O= ?( FLOAD TRIG) E.

:20) F: PI

Ft

F2* FSINCOS; (07:1 - -4

:)(EXP FSINCOS FROT FEXP X*F; g
(87. xy--e "xcos[y] e "xsixn[y])
:G(T) Z(T) XDUP 1/X XEXP ,

USE( G(T) % 0 %1 % 1.52 COMPLEX )INTEGRAL X. '

DEBUG- ON
N x F 1 E
0 0.0000 27102 0.0000 .50100 0.0000 0010900
1 .50000 -.3s707 0.0000 .50700 0.0000 mm

2 1.011113 2.7182 0.0000 .00000 .00000 .0000000
old.l = 2.7162 0.0000 finaIJ = OIXID 0.0000

N X F l E
0 0.0000 2.7182 0.0000 .50700
1 .SOGD -.$787 00000

0.0000 0040900
00190 .067537 .WW
2 75¢!” .84147 -.54030 .44496 -.007537 0024099
3 1.0000 2.7182 0.0m0 011100 .00000 .0000000
oIdJ = 587% 0.0000 finaIJ = OIXXD 0.0000

N X F I E

0 0me 2.7182 0.0111) 567% 0.0000 0049000

1 film-.5767 0.0000 .(591U-IE7537 £1124“
2 .7501) 34147-5403 .17“ -.oaasaz 0012400
3 .873!) 20219-15662 29020 4100014 001249
4 1.0000 2.7162 0.0000 .00000 000000 01110000
oId.I = .444” -.N7537 1Inll.l - 0.0000 0.00:»

N X F I E

0 0.0000 2.7162 0.0000 .5673) 0.0111) mm

1 SC!!!) 23787 0.0000 .W1U aWS'ST mean
2 .7511!) 34147-5401) .17” -.043002 .0012“
3 .875” 20210-15002 .1413) 413745 .00002400
4 00750 2519-0253 Team-name .tmezm
5 1mm 2.7162 0.01110 .ooamo .ootnoo .amm
Didi - maze-omens MALI - 0.0000 0.01110

 

N X F I E

0 0.0011) 2.7162 00000 56760 0.0000 noun

1 .50000-36707000113 .W1Q 457537 .112“ >'
2 .75000 54147-5461) .1790 mean .0312“

3 .8751!) 2.0219-.15& .14190-m5745 .m
4 33750 251$-.m .m1M-.m 00001240
5 .3875 2.6% -.00:10577 .06414 -.000055 11113120
6 MIX!) 27182 0.0000 .000000 .0000000 1111!!!!)
oId.| = Aw-IXDTBMZ fine“ = 0.0000 0.00m

N X F I E

0 0.01110 2.7162 0.0000 .5373) 0.0000 mama

1 50000-30707 0.0000 .050190-007537 mama
2 .75CKD .84147 -.s4m0 176$ -.oaaaa2 011249

3 .8751!) 2.0219-.15W 14190-000745: .m
4 .fl750 asm-mszza £0102 -.(ID4467 #1312“
5 .m 2.6% -.ooaasn 04197-101“ .IXXMW
6 .W 27052 010020 .042371 -.ommaa .tm1sa‘z4
7 1.0000 2.7162 0.0011) 0.0003 0.001!) 11mm

old.| s .004137-mm52405 finil = noun 001m

x F 1 E

0.0000 27102 00000 .50700 0.01100 0010000
.50000 -.30707 0.0000 000100 -.oe7rm mum
75000 54147-54000 .1701: -.0m2 001249
07000 2.0210-.15002 .14100 -.0m745 00002000 5
4 00750 20100-02022 .010000-.000a:10.00015024

0 30312 20000.011000041173-000112000010024

0 00075 2005-.00000001571-0011110001010024
0th - .001022-.000«007 mu - owns-mm

N
0
1

2
3

N X F I E

0 cm 2.718 can: mm mm .0312.

1 12m 2.0210 .1092 .1075! .0164" .m

I .167” 1.41” m M711 01m ammo

I 21675 1.1fl4 .aaaan m .0157” ammo

4 m .4147 54GB mm 0010070 0mm

“I - m mum Hill - 51W—m

N X F I E

0 cm 2.71” 0.00:!) as: £1.13. .m12“
1 .12” 2.0216 .15” mum 01101200
I .1” 1.7232 ms .04“ m.m1240
3 .10700 1.4M 3073 name 01578 01101240
ddl - .1079 916479 W1 - .3374 4mm

N X F l E

0 0m 2.7162 om film 00073421100de
1 .0023!) 2.51” 0329.141” 1157453 0003200
2 .12“ 2.0210 .1532 04011: 00300300001249
ddl - 29m 0mm fin.” - .fl13-1X55266

N X F l E

0 0001:) 2.71& cm am mam
1 mama 2.51“ mm 075m @133 00001240
2 M749 mmntmsrmasm .m1249
3 .125!) 2.0219 .15” moms 015/26 .m12fl
dfll - .141” ”7453 fnlll - $13 415528

N X F l E

0 0.0000 2.7182 00000 .1623 manuam

1 10m 251” m manna: M1249
2 mac 2.254 “75 134633.101“ .CXDIM
3 .1” 2.1032 .11“ “11.121“ 00015024

4 .1253 2.0219 .1902 m7m76 “1&4
de - M457 .0113!» Mall I miss-.maaa

52 Fltfing functions to data

 

N X F I E

0 cm 2.71” cm .1“ .W.m
1 M2510 .mm .mumm
2 07.182419047044m .mmcmstm
3 mommm.tm1m.mm
de - 075m 011“? Md! - .m-mtam

N X F l E

0 cm 27102001» new .m ammo

1 m1- 2.” W 1mm .M #1312“
2 W251“ .mm .MYLGXMW
ddl - .1” names: “I - ham-.m

N X F I E

0 00000 2.719 001m “4137 .m £11312“

1 mtao zen W7 .041173 M1247 00310024
2 0465752” .011“ mam.m.antasas
3 138132.510. 1330002110 .M‘Iflfi

oIdJ - 001022 .W “J - mas-00040523

N X F I E

0 00cm 2.7182 0.01113042371 m1 .CID1!£24

1 015% 27W m2 .041” 0001200 an 5624

2 .6313) 2.6” .W .osam m .m15fl4
oId.l - M4137 .m “J - 31$B<m6373
.0001» 90000001 oIr

\

 

Note:
answer = 1

 

 

 

One of the, most important applications of numerical analysis is
the representation of numerical data in functional form. This
includes fitting, smoothing, filtering, interpolating, etc.

A typical example is the problem of table lookup: a program re-
quires values of some mathematical function —sin(x), say — for
arbitrary values of x. The function is moderately or extremely
time-consuming to compute directly. According to the Intel tim-

oJuthNobIolm—Mmmarvod.

162

CMptarB-MoreProgmmnIngExamdac ScbnllflcFOR

 
   
  
   

ings for the 801187 chip, this operation should take about 8 tim
longer than a floating point multiply. In some real-time app '
tions this may be too slow.

There are several ways to speed up the computation of a functit
They are all based on compact representations of the fun '
-—either in tabular form or as coefficients of functions that a
faster to evaluate. For example, we might represent sin(x) by'
simple polynomial1

sin(x) ~x (0.994108 - 0.1472021) , (22) 1

,_
I
accurate to better than 1% over the range -% S x 5 g , thg'

requires but 3 multiplications and an addition to evaluate.
would be twice as fast as calculating sin(x) on the 80x87 chip1 .2

To achieve substantially greater speed requires table look-It]:

To locate data in an ordered table, we might employ binar3
search: that is, look at the x-value halfway down the table and se
if the desired value is greater or less than that. On the averag.
log2(N) comparisons are required, where N is the length of tla
table. For a table with 1% precision, we might need 128 entriet
i.e. seven comparisons.

 

Binary search is unacceptably slow — is there a faster methodl
In fact, assuming an ordered table of equally-spaced abscissae tlu
fastest way to locate the desired x-value is hashing, a method it!
computing the address rather than finding it using comparisom
Suppose, as before, we need 1% accuracy, i.e. a 128-point tablr’
with x in the range [0,:1/2]. To look up a value, we multiply: in
256/77 5 81.5, truncate to an integer and quadruple it to get 1
(4-byte) floating point address. These operations —includi1§
fetch to the 87stack— take about 1.5-2 fp multiply times, hencl
the speedup is 4-fold.

curl—MW!” 163

The speedup factor does not seem like much, especially for a
function such as sin(.r) that is built into the fast co-processor.
However, if we were speaking of a function that is considerably
slower to evaluate (for example one requiring evaluation of an
integral or solution of a differential equation) bashed table
lookup with interpolation can be several orders of magnitude
faster than direct evaluation.

e now consider how to represent data by mathematical func-
tions. This can be useful in several contexts:

o The theoretical form of the function, but with unknown param-
eters, may be known. One might like to determine the param-
eters from the data. For example, one mi t have a lot of data
on pendulums: their periods, masses, imensions, etc. The
penod of a pendulum IS given, theoretically, by

1/2
- a L&
r—(g) f(r’ ."") (23)

where Lis the length of the string, g the acceleration of gravity,
and f is some function of ratios of typical lengths, masses and
other factors in the problem. In order to determine g accurate-
ly, one generally fits a function of all the measured factors, and
tries to minirmze its deviation from the measured periods.
That is, one might try

1/2
21 1'11 mbob
7n: (Tl-1') 1+a—L‘+fl( . ) +... (24)

for the n'th set of observations, with g, (2, fl, the unknown
parameters to be determined.

0 Sometimes one knows that a phenomenon is basically smooth-
ly varying; so that the wiggles and deviations in observations
are noise or otherwise umnteresting. How can we filter out the
noise without losing the significant part of the data? Several
methods have been developed for this gurpose, based on the
same principle: the data are represente as a sum of functions
from a complete set of functions, with unknown coefficients.
That is, if 90.,(x) are the functions, we say (y. are the data)

C

Ya E 2 C. plan) (25)

m-o

OJLIHnVNobIatm—Alnoflaroaarvod.

Chapters-MoreProgrammlng Examples Schnfinc FORT

Such representations are theoretically possible under gene
conditions. Then to filter we keep on] a finite sum, retainin
the first N (usually simplest and smoot est) functions from th
set. An example of a complete set is monomials, ¢,(x) = x m
Another is sinusoidal (trigonometric) functions,

sin(2:rm.r), cos(27rmx), 0 s x s 1,

used in Fourier-series representation. Gram polynomials, dis
cussed below, comprise a third useful complete set.

The representation in Eq. 25 is called linear because the urts

known coefficients cm appear to their first power. Thus, if a”
the data were to double, we see immediately that the cm's would
have to be multiplied by the same factor, 2. Sometimes, as in the
example of the measurement of g above, the unknown paratr~
eters appear in more complicated fashion. The problem of fitting
with these more general functional forms is called nonlinear fon
obvious reasons. The simplex algorithm of Ch. 8 §2.3 below is an
example of a nonlinear fitting procedure.

 

We are now going to write programs to fit both linear and non—
linear functions to data. The first and conceptually simplest ofl
these is the Fourier transform, namely representing a function as
a sum of sines and cosines.

§§1 Fast Fourier transform
What is a Fourier transform? Suppose we have a function thati
is periodic on the interval 0 s x 5 2m

f(x + 2:) = to);

Then under fairly general conditions the function can be ex~
pressed in the form

f(x) = a0 + E (a, cos(n.r) + b, 5111011)) (26)

11-1

Another way to write Eq. 26 is

f(x) -:§c.ew. (27)

In either way of writing, the c,, are called Fourier coefficients of
the function f(x). booking, e.g., at Eq. 27, we see that the or-
thogonality of the sinusoidal functions leads to the expression

1 _ .
c..=gf3'f(x)e "“411. (28)
Evaluatin£ Eq. 28 numerically requires —for given n- at least
2n pomts Naively, for each n = 0 to N-l we have to do a sum
2N

~ 2 f1. e—mk/N

k-l

C11

which means carrying out 2N2 complex multiplications.

The fast Fourier transform (FFT) was discovered by Runge and
Konig, rediscovered by Danielzslon and Lanaos and m—redis-
covered by Cooley and Tukeyn. The FFI‘ algorithm can be
expressed as three steps.

0 Discretize the interval, Le. evaluate f(x) only for
k
27: N , 0 S x S N 1.
Call f(x.,) = f.,.
0 Express the Fourier coefficients as
N —1
= 2 r. (2.....- ”". (29)
11-0
0 With w, = e"2"“"'/N , Eq. 29 is an N-l’st degree polynomial in
w, . We evaluate the polynomial using a fast algorithm.

 

20. to prevent aliasing.
21. See, e.g., DE. Knuth, Thain of Computerhogmnmting v. 2 (Addison-Wesley Publishing Co.,
Reading, MA, 1981) p. 642.

OJtlthNohlotm—Alrlgflamennd.

Chapter 0 - More Programming Examples Scientific FORD-Ir

To evaluate rapidly the polynomial

N-l
c. = P. (w.) a 2 1.011.)"

k-O
we divide it into two polynomials of order N/Z, dividing each of]
those 1n two, etc. This procedure' IS efficient only for N= 2 ,with,
v an integer, so this IS the case we attack.

How does dividing a polynomial in two help us? If we segregate:
the odd from the even powers, we have, symbolically,

Paw) = EN/2(W2) + w omw’) . (30)

 

Suppose the time to evaluate PN(w) is TN. Then, clearly,

..4._-..~.<-Aanv- .. ‘1

TN = A '1' ”w; (31)

where .1 is the time to segregate the coefficients into odd and even.
plus the time for 2 multiplications and a division. The solution of’
Eq. 31is 1(N-1). That is, it takes 0(N) time to evaluate a poly-
nomial.

However, the discreteness of the Fourier transform helps us here.
The reason is this: to evaluate the transform, we have to evaluate

PN (W) for N values of w, . But wfi takes on only N/Z values as 11
takes on N values. Thus to evaluate the Fourier transform for all
N values of n, we can evaluate the two polynomials of order NB
for half as many points.

Suppose we evaluated the polynomials the old-fashioned way: it
would take 2(N/2)= -N multiplications to do both, but we need do
this only N/Z times, and N more (to combine them) so we have
N 2/2 + N rather than N2. We have gained a factor 2. Obviously 1t~‘
pays to repeat the procedure, dividing each of the sub-polyno~
mials in two again, until only monomials are left.

Symbolically, the number of multiplications needed to evaluate,
a polynomial for N (discrete) values of w is s

1,. = N1 + 21",, (32) §

 

MO-MPWEW 107

whose solution is

1,, - AN 103,01) . (33)

Although the FFT algorithm can be programmed recursively, it
almost never is. "lb see why, imagine how the coefficients would
be re-shuffled by Eq. 30: we work out the case for 16 coefficients,
exhibiting them in Table 8-1 below, writing only the indices:

Table 8-1 Blt-reverad tor reordering discrete data

 

Stan Stqfi Stop2 SW3 Bho Bhg
0 0 0 0 (XXX) (XXX)
1 2 4 6 (XXII “XX!
2 4 6 4 N10 0100
3 6 12 12 (D11 1100
4 B 2 2 0100 NIO
5 10 6 10 0101 1010
6 12 10 6 0110 0110
7 14 14 14 0111 1110
0 1 1 1 1000 (”01
9 3 5 9 1001 1001
10 5 9 5 1010 0101
11 7 13 13 1011 1101
12 9 3 3 1100 (1)11
13 11 7 11 1101 1011
14 13 11 7 1110 0111
15 15 15 15 1111 1111

 

 

The crucial columns are “Start” and “Step 3". Unfortunately, they
are written in decimal notation, which conceals a fact that be-
comes glaringly obvious in binary notation. So we re-write them
in binary in the columns Bino and Bin, —and see that the final
order can be obtained from the initial order simply by reversing
the order of the bits, from left to right!

standard FORTRAN program for complex FFI‘ is shown below.
e shall simply translate the FORTRAN into FORTH as ex-
peditiously as possible, using some of FOR'IH's simplifications.

ne such improvement is a word to reverse the bits in a given
integer. Note how clumsily this was done in the FORTRAN

OJtlthNoUotm-Aldflttamorwd.

188

sumounre Foum (DATA. NN. ISIGN)
NunHueudunnnmdfil*nmbquau

ISBN EETEFNI'ESWl-EH'EHTl-EFFT
ISFWAHDORBAOKWAH)

0000000

DATA Is TT-E ((1)111le ARRAY GDISCRETE N’U‘I’
CDMPLEX w, WP. TEMP, DATA(N)
DEAL'B TFETA
J =0
00 11 |= 0,N-1
IF (J.GT.I) THEN
TEMP= DATA“)
DATA(J) =DATA(I)
DATAtI) =TEMP
ENDIF
M: NR
1 IF ((M.GE1)AND.(J.GT.M)) THEN
J=J—M
M= W2
60 T0 1
ENDtF
J = J + M
11 CONTINUE

\boainmmnu

\endbitrwerul

Chapter 3 - More Programming Examples

 

Scientific FORTH

WAX-1 \mmm
2 F(MGT.MIMT|-EN
STE-m \Mumm

THEM-ammmmm
WP-CEXPO'HEI’A)
w- Window)
0013 M- 1.1mm
no 121- M.N.ISTEP
J=1+MMAX
TEMP=DATA(J)'W
mm» =DATA(l)-TEAP
mm» = 0mm) new
12 oomwue \end lnnerloop
c

\mbov
\lmerloop
\utII-Ntlrnu

W=W‘WP \tng m

C

13 CONTINUE
MMAX=ISTEP
GOT02
BJDIF

FETUFN

\endoubrloop

\endMeleon—Unazoeudon

program. Since practically every microprocessor permits right-
shifting a register one bit at a time and feeding the overflow into
another register from the right, 3.11 can be programmed easily in
machine code for speed. Our fast bit-reversal procedure 8.11 may
be represented pictorially as in Fig. 8-5 below.

 

Initial number: Reg A
:1001101110100011-1

    
 

 

 

Stepv
III a

 

Step 1

Bit-reversed number: Reg 3

1100010111011001

Bit overflow

”1W1“1“6°’ii1“6‘1o°66‘i ‘

'#AWA“V’»Y~V¢WIMWW“‘<%MVWW ‘.

Inltlel number: Reg B

 

 

 

 

 

Fig. 8—5 Pictorial representation of bitmrsel

DWI—WWW 1”

Bit-reversal can be accomplished in high-level FORTH via

:B.R (n--n') \reverseorderoibits
OSWAP (--On) \eetupstack
N.BITSODO

DUP1AND \pickout1'sbit

ROT2'+ \leftshift1,edd1'sbit

SWAP2/ \rightoshittn
LOOPDROP;

I Note: N.BITS is a VAR, previously set to v = log2(N)

We will use 3.8 to re-order the actual data array (even though
this is slightly more time-consuming than setting up a list of
scrambled pointers. leaving the data alone). We forego indirec-
tion for two reasons: first, we have to divide by N (N steps) when
inverse-transforming, so we might as well combine this with
bit-reversal; second, there are N steps in rearranging and dividing
by N the input vector, whereas the FFT itself takes Nlog2(N)
steps, Le. the execution time for the preliminary N steps is unim-
portant.

Now, how do we go about evaluating the sub-polynomials to get
the answer? First, let us write the polynomials (for our case
N = 16) corresponding to taking the (bit-reversed) addresses off
the stack in succession, as in Fig. 8-6 below.

 

 

:12; i 5233* w... + ...
w5(t,, + M 1,3)
30‘ + 5f” : >co + we,
fa + 114 4
Min + M 1,0; ”2“” + w 8")
w‘(r. + w’ in)
M0,, + M r. )1

>w‘(b, + wing)
w‘(a1 + w‘as)

) w°(bo + w’bz)

w°(a° + w‘a.)

 

 

 

J

 

 

Flgjel’heorderolevalwtlnqe 16m. FFT

cmvmrm-umm.

190 mus—Mmrrogmmnmanm SclentIflcFORTr/s

 
  

We see that w? (for N=16) has only two possible values, :1.

Thus we must evaluate not 16 x 8 terms like 1’, + w8 fm , but only
2X8. Similarly, we do not need to evaluate 16X4 terms of form
f, + 1941',“ , but only 4X4, since there are only 4 possible values
of w:. Thus the total number of multiplications is

=5fifiefifixwifi

2x8 + 4x4 + 8x2 + 16x1 = 64 a16logz16,

is!

as advertised. This is far fewer than 16X16 =256, and the ratio

improves with N — for example a 1024 point FFT is 100 times
faster than a slow FT.

We list the FFI‘ program on page 191 below. Since }FFT trans '
forms a one-dimensional array we retain the curly braces notation I
introduced in Ch. 5. We want to say something like

new.

V{ n.pts FORWARD }FFT

where V{ is the name of the (complex) array to be transformed,
n.pts (a power of 2) is the size of the array, and the flag-setting ; '
words FORWARD or INVERSE determine whether we are
taking a PET or inverting one.

Now we test the program. Table 8—2 on page 192 contains the L"

weekly stock prices of IBM stock, for the year 1983 (the 52
values have been made complex numbers by adding 01', and the
table padded out to 64 entries (the nearest power of 2) with
complex zeros)22 .T'he first two entries (2,64) are the type and
length of the file. (The file reads from left to right.)

We FFl‘ Table 8- 2 using the phrase IBM{ 64 DIRECT }FFT. The
power spectrum of the resulting FFI‘ (Table 8-3) is shown in
Fig. 8- 7 on page 192 below.

 

22. This example is taken from the article “FORTH and the Fast Fourier Transform" by Joe
Barnhart, Dr. Dabb's Journal, September 1984, p. 34.

 

OJtllnnVNoble1m—Nldghuruorvod.

 

”FORTH

\WMMW
Kw:mmmmrmnm
.rAaxm
\mmmammww

MCs o-rmmooowrsx)

I'D THY O-‘HHDIDMATMHBFD
Mom o-rinvonaorm)
norm o-‘NFLOAD‘I'Ml

\IMM

m
\‘I..---------.---...-...-I
(mm

: oooeeaememanoooe

: moan-4.21111) oswAP (momma
3am NC»
was SH! MPH) 9V» REPEAT:

:CIN N$>FCJF:

15W 1150mm

; F ONOSWAP UN CPSNAP TIEN:
(era-mm

\u-ncu-------------.--nun-u

\byblmml

PM!!!

'2“ (mm) \memdbh
- new» (-—0n) \ltipdnk

1' NSTSOWMIADD \MMI'IUI

mew \Mmlflmi‘efl
MP2] \fl-m
' mm;

11521133501511:
Mono 13.11 ISLR
1 LRI<IOT (LR>-17)
“unmanned WEE
1(1A}o11.1(1}oitniat

(“Wflm

01.1.0-”me

 

191

\-.--...--.--...-.--------.
\mdnw
01204 ”10.160113”.
IFW.1W.3FLD,
1WJMJFWJWW
(Chop-H 1H2)

:Tl-ETA F-HMSoFF/ m M;

mmwwmmoaw
:mmo mmseommocnc-i;

:reww (arm-4')» 0170 c-;
OVARISTEP

zoomwov
00 m1 + 19111
cm 111111001- 0' 1(1) eat
WHO-<1» 1(1)01L1(111}GIL
1819 +L00";

:}H-‘T (uknn) ISN ISM

”SW

NLGZISMBTS

FNT

BITRBIERSE

BEGN
NM >

was
NTJ'FIGMMAXTISISTEP
W000

NImDNELW WWW

w
ISTEPISMMAX

RE’EATCPWP:

:mnooormamcmscal. F. w;
\mmdFF-‘T \mddmm
\-I-I----.---.----3-----.I-
\Inmeb

“WWMMAYM
:NTA Afl'BMBfmFILCImE-IPUT:
NI'TA
MMWCI'WFT

\dfld-nmeb

\-.--m-u-u-n-u-n-u-n-u-u-u-

ow do we know the FFI‘ program actually worked? The
simplest method is to inverse-transform the transform, and

compare with the input file. The FFI' and inverse FFT are given.

respectively, in Tables 8-3 and 8-4 on page 193 below. Within
roundoff error, Table 8-4 agrees with Table 8-2 on page 192.

emvmuueea-Almw.

Chapter 8 - More Programming Examples Scienfific FORTH

Table 6-2 Weekly IBM common stock prices, 1983

 

2 84

5.63 0.0 9.13 0.0 91.- 0.0 07.5 0.0
97.38 0.0 5.38 0.0 case 0.0 mm 0.0
11125 0.0 1&75 0.0 fl.fl 0.0 1&13 0.0
101$! 0.0 mas 0.0 110.13 0.0 117.5 0.0
117.00 0.0 117.63 0.0 116.3 0.0 110.5 0.0
113.11) 0.0 mm 0.0 114.5 0.0 121.13 0.0
123m 0.0 121.“) 0.0 121.50 0.0 15.13 0.0
124.38 0.0 15.38 0.0 119.75 0.0 118.50 0.0
15.5) 0.0 117.83 0.0 119.75 0.0 12.5 0.0
15.13 0.0 15.5 0.0 15.38 0.0 1325 0.0
131.75 0.0 127.00 0.0 15.1.!) 0.0 1&5 0.0
15.88 0.0 1235) 0.0 121.00 0.0 117.88 0.0
12.25 0.0 120.88 0.0 15.5 0.0 122.00 0.0

0.0 0.0 0.0 0.0 0.0 0.0 0.0 0.0
0.0 0.0 0.0 0.0 0.0 0.0 0.0 0.0
0.0 0.0 0.0 0.0 0.0 0.0 0.0 0.0

 

 

 

 

 

6000
5400
4800
4200
3600
r 3000
- 2400
- 1800
- 1200
. 600

L4._i

AALALA

       

 

     

8 43,

0.0 5.4 10.8 16.2 21.6 27.0 32.4 37. 2 48.6 54.0

 

 

Fig. 8-7 Power spectrum of FFT of 1983 IBM prices (from Table 63)

i

 

 

MNFFTdiBMweekiyetockpricu, 100:1

 

 

 

 

 

 

 

0 man 0m 02 01.11.00.113!!!» 107.m.1s1m40 40 72mm 10.171040
1.1.100- mm as amum 1727040402 01001004 40 nmmam
207400417 m 34 unnunm 10.134201 117m 01 140.1040104107007
0 011000114 54.40010 3 nmm 1011002031 71.4224u s1 ummmm
4 02741342 17701110 a ammou 2007” 24.00.74 02 40.01m14740120
3 12021042011301” 37 40102017314041442 21104070107340“ 0:1 40.470211042121100
um 9001013 3 1010021an mum 02.311140 04 00270041331000
700020402 name a swam 2044172011077001 00 10110401411900
0 1924a 200.01” 40 10771701044100" 2410771731 34.411177 :0 11124342001000
0 10110401 12015400 41 44.17me1107731 200073002 am 57 00.012040222027100
10012700414111.3103 42 mama-0201:1040 20102011210211.2001 00 .170m000013
1140.47sa1 0.4212000 4:1 10407913154041 27mm": 14.041442 00 13210120000304
‘ 1240mm 147.4513 44 07.301104242501174 20:0.700000 00.411000 :1 02740042177011 10
13045024001521202 40 11003017142402 2001700004 25.042037 01 30000314004010
1414010401 04.107037 40 «imam-11700700 3000170007 02:14am a 41742041740110.0422
mammam 47 47.114040201001004 31 13.411000744057210 0:1 Jason-0am
Table 8-4 Reconstructed IBM prices (inverse FFT)
o 000:!) 00000 :12 12250 0.11110 10117.00-000000000 40 1225 000000000
1 00.120 000100705 30 11702-110000245 17 117.02-000000510 40 12007 0011100132
2 04020 0000011023 31 110.75 000000140 10 11s.so-.000000110 so 12300000000002
3 07070 1100000227 :15 12224000000070 10110.02000001375 51 1213-0000000“
4 07000 0000000011 :10 1201000100005 20111000000010" :2 .000012004-mmo1070
5 00.370411100011302 37 12002 0000007110 21 114.00 000000075 00 -.000001277-0000014:14
e m m1a7 a 1203 001000051 22 1142; 000000070 54 -mcmzm-ma:o1224
7 1003 000002150 :10 1325-000101001 23 121.12 000000017 55 -.00m05000-.000000015
0 10254110000000 4o 131.70-000000000 24 1Z1000000001m 50 00000010120011an
0 10070 000000700 41 12700-001000073 25 12100 0110001130 57 -.000002005-000001000
10 0030 -.0m0m271 42 120.00 000000401 20 121.50-01mo1400 50 4100010040 100001300
11 10212-01mmra 40 12224-000001011 27 120.12 003012022 so -.000m5020-m0001711
12 10102 1110000010 44 1307 0000000111 20 124.37 .000000027 :1 -.m0000010-.mm00m7
13 100 001000043 45 121.50 mm 20 120.30 000001201 01 -.0000_:1-000001500
14 110.1:1-m1010 40 121.00 000001007 :10 110.75-001000010 02 -.000000121 000000200
115 nus-0.0.7 47 11707 00001104 31 11040 000000573 03 -.0000m713 000010000

 

 

 

§§2 Gram polynomials
Gram polynomials are useful in fitting data by the linear least-
squares method. The usual method is based on the following
question: What is the “best" polynomial,

N
P1111) = 207. r". (34)

omvmunm-umm.

Cl'teptere—MoreProgrammingExamplee Sclendiic FORTH ,

(of order N) that I can use to fit some set of M pairs of data points, 1

1

{1"}, k=0, 1, ,M—1 1
k

(with M > N) where f(x) is measured at M distinct values of the
independent variable x ?

The usual answer, found by Gauss, is to minimize the squares of
the deviations (at the points xk ) of the fitting function PN (x) from
the data — possibly weighted by the uncertainties of the data That
is, we want to minimize the statistic

2 M—1 N
X = 2 (ft _ 2 1’11in
k=0

2 1
“=0 ) 2 (35)

with repect to the N + 1 parameters yl| .
From the differential calculus we know that a function’g first
derivative vanishes at a minimum, hence we differentiate x with

respect to each yn independently, and set the results equal to
zero. This yields N + 1 linear equations in N + 1 unknowns:

2 Anm 7m =fln 1 "=011, , N (36)

where (the symbol 9 means “is defined by”)

A M_1 71+m 1
Anm = 20 (x11 ; (37a)
' k
and
A ”'1 1
fin = ongfk— (37b)

In Chapter 9 we develop methods for solving linear equations.
Unfortunately, they cannot be applied to Eq. 36 for N 2 9 be-
cause the matrixAm approximates a Hilbert matrix,

 

const.
H“ - n+m+1 ’

MI-MWW 195

a particularly virulent example of an exponentially ill-
eondltlonetl matrlx. That is. the roundofl' error in solving 36
grows exponentially with N, and is generally unacceptable We

can avoid roundolf problems by expanding' in polynomials rather
than monomials:

2 g 2 i
z E. (1. lg) 7.12.0.1) of 138)
The matrix then becomes
14— 1
=20 pn(xk)pm(xk)_ of (39a)
and the inhomogeneous term is now
M-1 1
fin = 2 17110:)ka (39b)
k-o 1:

5 there any choice of the polynomials p,,(x) that will eliminate
roundofi'? The best kinds of linear equations are those with
nearly diagonal matrices. We note the sum in Eq. 39a is nearly an
integral, if M is large. If we choose the polynomials so they are
orthogonal with respect to the weight function
W(x) =fi01x. - x) 0(x — ..-),
k

where

0, <0
0(x)={1:20

thenA“, will be nearly diagonal, and well-conditioned.

196

Chapter 3 - More Progmmmhg Examples Scientific FOR!

Orthogonal polynomials play an important role in numerical an:

y%s and applied mathematics. They satisfy (Ithogmality rat
tions of the form

B _ = 1, m=n
I,drw(x)p..(x)p..(x) — a... — {0, ,m (40)
where the weight function w(x) is positive.

For a given w(x) and interval [A,B], we can construct orthogom
polynomials using the Gram-Schmidt orthogonalization process

Denote the integral in Eq. 40 by (p, , Pm ) to save having to writ
it many times. We start with

p—l = 0 )
-l/2
P007) = (fi dr W(x)) = const. ,

and assume the polynomials satisfy the 2-term upward recursia
relation

Pn+1(x) = (an + xbn ) pn(x) + cup..-1(x) (41)

Now apply Eq. 41: assume we have calculated pn and pr, an
want to calculate 17,“. Clearly, the orthogonality property give:

(Pn+1.Pn)=(Pn+hPn-1)=(Pn:Pn-1)=0a

and the assumed normalization gives

(p...p..)=1-

These relations yields two equations for the three unknowns, 0
bn and C n:

 

23.

Polynomials can be thought of as vectors in a space of infinitely many dimensions (“Hilbert'
space). Certain polynomials are like the vectors that point in the (mutually orthogonal) direc.
tions in ordinary 3-dlmensional space, and so are called «diagonal by analog.

awe-mew” 191

a. +0.0». xp.)-o
c. + b. 02.. xp.-.) - 0
We express a. and c, in terms orb, to get
p...(x) = b.[(x - (pi. xp. )) p.(x)
- 02.. xp.-. mam]

We determine the remaining parameter b. by again using the
normalization condition:

(42)

(pn4-1 v pn+l ) = 1

In practice, we pretend b.l =1 and evaluate Eq. 42; then we cal-
culate

b. = Gm. 5.... )‘W. (43)

multiply the (un-normalized) 1-3,,“ by b, , and continue.

The process of successive orthogonalization guarantees that p. is
orthogonal to all polynomials of lesser degree in the set. Why is
this so? By construction,p.,+1.l.p.. and p,,,. Is it tp, .2 ? We
need to ask whether

(pn v (x — an)pn-2) = 0-

But we know that any polynomial of degree N-l can be expressed
as a linear combination of independent polynomials of degrees
0, 1, . N-l. Thus

n-l

(at-amp“E 2 supra) (44)

x-o
and (by hypothesis) pI .L every term of the rhs of Eq. 44. hence
it follows (by mathematical induction) that

pa” J- {pn-thn-Si }.

OJUMVNohletm-Almm.

Were-Moro Programming!“ Scientific FORTH

Let us illustrate the process for Legendre polynomials, defined
by weight w(x) = 1, interval {-1.1}:

_ 1 1/2
PO" 2 I

..................................................................

These are in fact the first three (normalized) Legendre polynom-
ials, as any standard reference will confirm.

Now we can discuss Gram polynomials. While orthogonal poly-

nomials are usually defined with respect to an integral as in
Eq. 40, we might also define orthogonality in terms of a sum, as
in Eq. 39a. That is, suppose we define the polynomials such that.

”—1 { 1, m=n

kgopn(xk)pm(xk);:; E dnm = 0, matn (45)

Then we can construct the Gram polynomials, calculating the
coefficients by the algebraic steps of the Gram-Schmidt process.
except now we evaluate sums rather than integrals. Since p..(x)
satisfies 45 by construction, the coefficients y, in our fitting
polynomial are simply

M-l 1
7.. = E p..(xt) fig; (46)
k-O k
they can be evaluated without solving any coupled linear equa
tions, ill-conditioned or otherwise. Roundoff error thus becomes
irrelevant.

The algorithm for fitting data with Gram polynomials may in
expressed in flow-diagram form:

neodhpomt...x.¢mw. 1.1/0.3.
DOn-ttoN-t (outerloop)
Construct an , cn :
DO k=o to M1 (inner loop)
pn+t(xlr) = (XI: " 8n)Pn(Xu) - crpn-1(Xtr)

sum = sum + (p0,,(x.)) 2w.

Com = an + ft Pn+1(Xk) Wk
LOOP (and inner loop)
cn+1 = cn+1 /sum

DO k =0 to M-1 (normalize)
’ Pn+1(xk) = pn+1(xk)/ mm
LOOP

LOOP

 

 

Fig. 8-5 Construction ofGram polynomials

The required storage is 5 vectors of length M to hold xb p,(xk),

p.,.1(.\:.,),f.l and Wu -:- 1/03 . We also need to store the coefficients
an, c. and the normalizations b. —that is, 3 vectors of length
N < <M— in case they should be needed to interpolate. The time
involved is approximately 7M multiplications and additions for
each n, giving 7NM. Since N can be no greater than M-l (M data
determine at most a polynomial of degree M-l), the maximum
possible running time is 7M2, which is much less than the time to
solve M linear equations.

In practice, we would never wish to fit a polynomial of order

comparable to the number of data, since this would include the
noise as well as the significant information

emvuouoim-me.

200 ChapterB-MoreProgrammthxamplee SclentlflcFORTH

We therefore calculate a statistic called zz/(degrce of freedom)“:
With M data points and an N’th order polynomial, there are
M-N-l degrees of freedom. That is, we evaluate Eq. 38 for fixed
N, and divide by M-N-l. We then increase N by l and do it again
The value of N to stop at is the one where

2
= 1ng
”'2“ M-N-l

stops decreasing (with N) and begins to increase.
The best thing about the xfiN statistic is we can increase N without
having to do any extra work:

2 M-1

ZM,N = 2 (ft -§ l’nPn(xk))2Wk

- (47)
M-l

N
a 2 «oz — 2 (7..)2
k=0 n=0

The first term after E in Eq. 47 is independent of N, and the
second term is computed as we go. Thus we could turn the outel
loop (over N) into a BEGIN WHILE REPEAT loop, in

which N is incremented as long as of”; is larger than 02”“ .
(Incidentally, Eq. 47 guarantees that as we increase N the fitted
curve deviates less and less, on the average, from the measured
points. When N = M-l, in fact, the curve goes through the points
But as explained above, this is a meaningless fit, since all data
contain measurement errors. A fitted curve that passes closer
than 0., to more than about 1/3 of the points is suspect.)

The code for Gram polynomials is relatively easy to write using
the techniques developed in Ch. 5. The program is displayed in
full in Appendix 8.4.

 

24. That is, “chi-squared per degree of freedom".

MFOR‘M

DWI-MW!!!“ 201

WWW

ometimes we must fit data by a function that depends on
parameters in a nonlinear manner. An example is

F
‘- 1—7777) “3’

Although the dependence on the parameter F is linear, that on
the parameters a and X is decidedly nonlinear.

One way to handle a problem like fitting Eq. 48 might be to
transform the data, to make the dependence on the parameters
linear. In some cases this is possible, but in 48 no transformation
will render linear the dependence on all three parameters at once.

Thus we are frequently confronted with having to minimize

numerically a complicated function of several parameters. Let
us denote these by 00, 01 , , BN4 , and denote their possible
range of variation by R. Then we want to find those values of
{ 0}C R that minimize a positive function:

12 (so, ,UN_,) 101;?" 12(00,...9N-,) (49)

One way to accomplish the minimization is via calculus, using a
method known as steepest descents. The idea is to differentiate
the function 12 with respect to each 0,, and to set the resulting N
equations equal to zero, solving for the N 0’s. This is generally a
pretty tall order, hence various approximate, iterative techniques
have been developed. The simplest just steps along in 0-space,
along the direction of the local downhill gradient -sz, until a
minimum is foun . Then a new gradient is computed, and a new
minimum sought .

Aside fromthe labor of computing -V12, steepest descents has
two main drawbacks: first, it only guarantees to find a minimum,
not necessarily (In minimum — if a function has several local

 

25. maubywmyuemuwmdifiadmmbetoundinhmad,mmnm,

bid, p. 1113’.

Chapters-MoreProgramnIngExamuoa SclentIflcFOR‘m

I

I
rninima, steepest descents will not necessarily find the smallest.
Worse, consider a function that has a minimum in the form of a
steep-sided gulley that winds slowly downhill to a declivity —
somewhat like a meandering river's channel. Steepest descents
will then spend all its time bouncing up and down the banks of
the gulley, rather than proceeding along its bottom, since thel
steepest gradient is always nearly perpendicular to the line of thei
channel. 1‘
Sometimes the function 12 is so complex that its gradient is tooI
expensive to compute. Can we find a minimum without evaluating}
partial derivatives? A standard way to do this is called the sins?
plex method. The idea is to construct a simplex — a set of N +1
distinct and non-degenerate vertices in the N-dimensional 0-
space (“non-degenerate" means the geometrical object, formed
by connecting the N + 1 vertices with straight lines, ha non-zero
N-dimensional volume; for example, if N=2, the simplex is a
triangle.)

We evaluate the function to be minimized at each of the vertices,
and sort the table of vertices by the size of 12 at each vertex, the
best (smallest x2 ) on top, the worst at the bottom. The simplex
algorithm then chooses a new point in 0-space by the a strategy,
expressed as the flow diagram, Fig. 8-8 on page 203 below, thd
in action somewhat resembles the behavior of an amoeba seeking
its food. The key word )MINIMIZE that implements the complex
decision tree in Fig. 8-8 (given here in pseudocode) is

: )MINIMIZE (n.'rter - - 87: rel.error - -)

INITIALIZE
BEGIN done? NOT NN.max < AND
WHILE

REFLECT r> =best?

IF r> =2worst?

IF r<worst? IF STORE.X THEN
HALVE r<worst?
IF STORE.X ELSE SHRINK THEN
ELSE STOREX THEN
ELSE DOUBLE r> =best?
IF STOREXP ELSE STORE.X THEN
THEN
N 1 + IS N SORT
REPEAT ;

summer-M Wat-MW!” 203

usedintheformat

USE( tnarne 20 96 1.6-4 )MINIMIZE

Fleshing out the details is a -by now— familiar process, so we
leave the program perse to Appendix 85. We also include there
a FORTRAN subroutine for the simplex algorithm, taken from

 

 

 

INHIALIZE \ choose simplex? is am of param. space
BEGIN
Converged? NOT N N.rnax s
WHILE
'—- REFLECT\ worst point thru geoeenter of other points
DONE '0") S «best-Pt) 7
.~
I
Yes No l(x') z l(2worst.pl) ?
DOUBLE '/\’
N Y
xx") s «bestpo ? m r ’0”) < “3”“ 7
l .

Am REPEAT .
Store r'

HALVE
i
f(x") < I(worst.pl) ?
REPEAT A

Store r' SHRINK

REPEAT

 

 

Fig. 0-3 Flow dlagram of the slmplex algorithm

OJUthNohletm—Almraaervod.

204

Ghana 3 — More Programming Examples Scientific FORTH

Numerical Recipes 26, as an example of just how indecipherable
traditional languages can be.

53 Appendices

§§1 Gaussian quadrature

aussian quadrature formulae are based on the following idea:

if we let the points 6,, and weights wll be 2N free parameters
(n runs from 1 to N), what values of them most accurately repre-
sent an integral by the formula

B N
I = [A dx a(x)f(x) = E1 w,,f(§,,) ? (50)

In Eq. 50 0(x) is a (known) positive function and f(x) is the
function we want to integrate. This problem can actually be
solved, and leads to tables of points 5,, and weight coefficients
w,, specific to a particular interval [A,B] and weight function a(x).
Gauss-Legendre integration pertains to [-1,+1] and o(x)= 1.
(Note any interval can be transformed into [-1, + 1].)

The interval [0,00) and 0(x) = e"r leads to Gauss-Laguerre for-

2
mulae, whereas the interval (— oo,+ 00) and 0(x) = e”Jr leads to
Gauss-Hermite formulae.

Finally, we note that the more common integration formulae such
as Simpson’s rule or the trapezoidal rule can be derived on the
same basis as the Gauss methods, except that the points are spe-
cified in advance to be equally spaced and to include the end-
points of the interval. Only the weights w, can be determined as
free fitting parameters that give the best approximation to the”
integral.

For given N Gaussian formulae can be more accurate than equally-
spaced rules, as they have twice as many parameters to play with.

 

26.

Press, et a1. , Numerical Recipes, laid... p. 29“.

mum Owe-MW!” 206

Some FORTH words for 5-point Gauss-Legendre integration:

 

 

N swarm WANT fl
1. ossamroreasss FCONSTANT x1 :mm (arse--0
it ow Pomsrmr wo soda (arse-[mew [sat/2)
N 0.4mm FWANT wt FCNER Fm IOF' F-ROT (87:-Jab)
% QWID WTANT W2 XDIP in” FIX) wt F' F3R+
XDIP x1 FPEGATE rude
:acde (arse-[mam [eat/2) Poo m F‘ F3R+
FOVERF—leFtNDERF+; Mia“ F(X)\~2F' F$+
:reeede (a7:abx--a+b"x) F'F+; XDUP )eFNEGATEreaede
':F3R+(e7:abcx--a+xbc)F3RF+ F-ROT; Pot) waF' F3R+
“2 The trapezoidal rule
ecall (Ch. 8, §1.1) how we approximated the area under a curve
Rby capturing it between rectangles consistently higher- and
lower than the curve; and calculating the areas of the two sets of
rectangles. In practice we use a better approximation: we average
the rectangular upper and lower bounds. The errors tend to
cancel, resulting m
B- A w
w ( 2y (f(A +nw) + f(A +nw+w))
n-O
(51)
-ff 2drf(x)+é(¥) w3max lf”(x)|
A<r<B
Now the error is much smaller28 - of order w2 — so if we double
the number of points, we ecrease the error four-fold. Yet this
so-called trapezoidal rule requires no more effort (in terms of
the number of function evaluations) than the rectangle rule.
27. See, eg, Abramowitz and Stegm, HMF, p.885.
‘28. we) istheaecondderivative offlx),l.e. the first derivative are).
29. Calledaobecausetheeuwejkfisapptofimatedbystraight lineaepnentsbetweensuecessive

printer": +w.‘l‘husweevaluatetheareasoftrapemidsratherthanrectangles.

OWVNohietm-Almmarved.

Ctnpter 8 - More Programming Examples Sclentlllc FORTH

“3 Richardson
In the program )INTEGRAL in Ch.8 §1§§5.3 the word INTER.
PO LATE performs Richardson extrapolation for the trapezoid»
a1 "rule. The idea' is this. If we use a given rule, accurate to order
w’I ,to calculate the integral on an interval [a,b], then presumably
the error of using the formula on each half of the interval and
adding the results, will be smaller by 2'". For the trapezoidal rule,
n =2, hence we expect the error from summing two half-inter-
vals to be 4x smaller than that from the whole interval.

Thus, we can write (10 = I: , I'o = I£a+by2 , I, = flaw/2 )
I0 = Iexact + R (523)

Equation 52b is only an approximation because the R that ap-
pears in it is not exactly the same as R in Eq. 52a. We will pretend
the two R’s are equal, however, and eliminate R from the two
equations (8.52a,b) ending with an expression for I...“ :

1...... ==1§(I’o + I. - 1.) (53)

Equation 53 is exactly what appears in INTERPOLATE.

MMTN

Owe-MW”

us MW: Orampolynomtata
: Here is a FORTH program for implementing the algorithm
I derived in {2552 above.

WON‘PI) ‘m. H348. 2RD. CM.

2” 1m.

DSDXKN.IXPOP. m
l‘Duel-47:eawagln]o[n-t]--aahw|g[n]g[n+1])

“zoos 804+!) 2FXO-I. 2941. an”.

1W1 1M1. DXDSWV. NR)!

1%: Demo m.m1[DK]FSTP.

DSOXWIBXPOP. 30m
Ind",
I7:aawag(n]g[n+1]~-aabwg[n+1]wxg[n+t])
in; mm) 1m oxosuov. OSPOP.
1 transmutation.
} MMFSTP. DSDXMOJ.BXPCP.W
I”db-87:anharg[n+1]vno[n+1]--aabwg[n+1])
I;DDEC(N+1)19012MJL'.2ML

2901 Hill”.

DXDSWV. DSFOP.

FBIDSIDQFAOD. DSIDQFSTP.

DSDXWV. BXPOP. SPAM. ENDOOII
lapel--w:aabwg[n+1]l-- a-a+wp[n+1]"2 a h)

I”:A{B(C(G((Mnax}Flf

‘FIISTAB’I F-o a(0}6! F-0b(0}GI;
NILE.“ (07:--g{{tl}}) F-o F-O
M000w(l}wy{l}G@ P'z FOUERF'
(Brea'uaa'wafl‘zt
HOT F+ HIST F+ Fat/AP
(87:--a-s+w a’-s‘+wt"2)
m an G! ream t/F;

 

:mm'e (wrelttlilnall I'll)
M000 r-o gunner PM slinnor LOOP;

2mm (flzgflllnuofllin)
F-0h110)01 mm
F-o mono-(iomo-(onoor- n LOOP
Pa(10)01;

:msraseoovoce (arm 11”.-)
F-Oe(00)0l F-o
mono wuoioo memo F' n uOOP
F'e(10)0l;

:msz Pun ISNnax sou 13cm“ tau
FRSTAB'. urnELrA mares seoowae.
masecouoc- ;

OVARN OVARN+1

:hc.N N+1|1PBN 1+ SN-H;
:m RWWGG R;
:thFF lBYTESW-i:
:ZERDL NF-OFBOIL;

:STAHTJbILG

(--[e{n+1)][a{n+1)][b(n+t}] 87:--a-0a{n} b{n})
Flth F-o e{N+tO}W ZEFOL
a{ NO} USPOSE mos? ZEDL
M NO} DISPOSE hem ZEXDL ;

:SEI’.FSTACK w{I‘0}G@x{l'0}G@ sIINfi‘rGG
aiiNl-I'IIGO:

:)@'l (linul'kx-q) HIP 0) usvose F‘Gl:

:Nomuuzs (”nun")
1/Fa(N+t)@'l F30"
b{N+1}@'i c{N+1}@-1
M000 FMQfiNHli) DISPOSE F‘ GI LEDP
m;
mm G'T'OPCKOPUOPCK
GPCKOPCKOPOK'W
mmmmmm‘m

Chapter 3 - More Programing Examples Sclerm‘llc FOR‘I

\ GRAM “MIL LEAST-0.1m W0)

Nmo smrmme
Honour SETFSTACK
“(NHIMDROP G(N+1) emu) A(N+t)
Yi'olGG GIN“)
LOOP COROP FDFOPFDFDP W;

:M.WLTA (z: — oid.delu muslin)
DELTAG@ FDJP e{N0)G@ F"2 F- F”
MTAGI;

:NOT.ENUF.G‘s? New.DELTA MN+1- S—F
FDUP FatF-

(CR .FS .' FEXTITERATDNT‘IYN Os FWTl-EM)
(:z-dd'm-n—t urn-2)
FFDT F\ F-RDT F/ (z: —d'l[m-n-2] d/[m-n-tl)
FOVER FOIFFESEFNDPFDROPOTFEN:

:}FIT (X{Y{S{ M‘naxM B{ C{G({ —)
NITIALIZE 1|SN 2ISN+1
BEGIN NOT .ENUF.G's’l NMnax AND
WHILE NextG inc.N
FEPEAT FINIT:
\ ------------------------------------- endoioode

:REOONSTRUCT MODOCRx{IO}G@F.y{I0}G@F.
F=ON+11+100
°{'°}G@GI{IJIIG@F‘F+
LOOP F.

LCXDPL

§§5 Non-linear least squares: simplex method
FORTRAN program for the simplex method is given belt
on page 209. The FORTH version, as discussed in §2§§3
given on pages 210 and 211.

WM?”

mm: mxmmmmmm
mm m- mm - 1.0.
: ma-oam-um-m
um mmvm.m.m.m
one - m H
nan-o
LO- 1
mutorvmner
"-1
m-a
as
Iii-2
N-I-t
EMF
no 1 1 I- LIFTS
scrotum) m-I
FlYmfltYII-QITI-Bl
M -H
"a!
ESE F(Y@.GT.YM)THBI
FGJEH "11-1
M
t W
RIO!- - Z‘WMYII-OMABSWMI + ABSMI-OIII
WLTFTGJETUFN

mam FALSE 'MM Mum Iterations.‘

Ila-ITS!“
emu-1.10M
Mil-0.
I W
Niel-1m
FWD-Q1184
no 13J- LION
w-wwo,»
m

EMF
.4 commas
i emu-1.10.1
Pam‘s-W
m-(1.+uhw-Pm.nmm.n
5 comes:
Vila-mm
mvmmea
emu-1.1a:
m-emmuromw
e cornea:

MI-MWW

 

run-mm
rmuxmnar
cows-um
Plug-emu)

17 W
veg-m
a!
Dona-1m
I’m-I'M

1s courses
vise-m
Ems
mrmoevmma
rmuxnmeu
cowl-1.1004
I’m-PRU)

10 course:
YM-YPR
error
mam-rm
Paar.» - BETA'Ptl-IJ) + (1 .Em-Pam.»

21 comma
YPm-FIMM
rmu.vm)m
00224-1.“
PMJ-W

22 OONTNUE
van-m
ELSE
oozu-uns
raNELotnar
oozes-1mm
W-WHJ+PILOJI
Pris-PR1.»

a courmE
Ym-HNKIPR)
error

24 courses
Blur

ELSE
ooasJ-mou
I’m-PRU)

25 comm:
wee-rm

ans

eer01

an

210

\WIIZATW BYTHESMPLEXIETHCD
\mmmstw Alan/1991

TASK MBA
\ ----------------------- WM MAW
VARABIE < F>
:USE( [oowusl ' CFA <F> I;
: F00 EXECUTE@ ;
BEI-EAD' <F>
\ -------------------- END FUNCTm NOTATm
\ -------------------------- DATA WES
3 VAR Him
0 VAR N
0 VAR N.mlx

CREATESIMPLEX{{ Ndim 4 (bytes) ' Ndim1+ (#pclnta)
'ALLOT

GREATER Ndimt+ 4 (bytes) ' ALLOT \nelduale

CREATEindex MimH- 2‘ ALLOT \arruyloraem'nbledlndleea

:>index (1—1') Z‘index+ @;

DVARIABLEW

DVARIABLE Fhaidual'

DVAFIABLEEpaIlon

CREATEX{ Ndlm 4(bytee)‘ ALLor \trialpoint
CREATEXH Ndim 4(bytea)‘ ALLOT \Z'Idlrialpolnt
CREATEY{ Ndirn 4(bytea)‘ ALLOT \geocenter

:} (adn—adr+4n) 4'4»; \partolarraynotation
2}} (adrrnn—adr+[m'NJirn+n]'4) SWAP Mim' + };

\ -------------------- ENDDArasrRucrunEs
\ ---------------------------- ACTIONWORDS
:FESIUJALS
Ndirn1+ 0 DO SIMPLEX{(IO}} Foo F{I} R32I
LCDP:

:<indax> Mim1+ODOlindexlr+ leP: \fllindax

:GIER <indox>
Ndm1+ooo F{I>1ndox}ma@
l1+BEGNNdlm1+ OVER
HERO/Bil >index}Fm@ FOVERFOVB! F>

FFSWAP
|>indettOVER >index
haxl2'+l momenta-+1
TI-ENFDFW 1+
RHEAT WW
LOO’;

CENTE(—) FNTMmSF
WOMF-O
WODO

(87:—Mm)
\Ioepwereomponents
\uraoeoverm

mums-11.1)} mm H
LOOP FOVERF/

Chapter 3 - More Programming Examples

 

Scientific FORT!

11mm

LOOP Ponce;
\m:mm1umumm4n
\--enehsdedl

\WII'I'Y

:OGIEI CR .'We'rellniaed.' ;
:TOOMANY CR .'Toomanyihratlona' :

:V.MOVE (arcadr doatadr-) \rnovevemr
NdimoDO OVER!) OVBIUZWIE
LOOP DOFDP:
:STORE (adriadr2--) O}
SIMPLEX({ NdmlndexO}}V.MOVE
F{Ndlrnlndex}2WVE;
:SI'OREX Fhaldul X{ STORE;
:STOREXP Reaidual'XH STOFE:

:New.F x{ Foo Fbaidud F1321;
:EXTRUDE (B7:aeale.laetor--)
\exbndpaeudopod
Murmo no DUP 1} new (87:-4.1x)
Y{I} m FUNDEI F- (37:--e.fyxy)
FROT FUNDER F“ (B72--yr~llx-yl‘l.o
FFOT F+ (57:--s.ly+[x+/]fi.l)
X{l} mar
LOOP DFCP W MI;

:F=1/2 F=1FNEGATEF=1FSCAI£FPLLCI(:
:F=2 F=1FDUPFSCALEFPIJEK;

\ ---------------------- DEBUGGNGCOE
OVARDBG
:DEBUG-ON-tlSDBG:
:DEBUGOFFOISDBG:
:.V SSPACES Ndimo DO DUPI} WP.
LOOPDFOP;
:.M (--)
Ndimt+ODODBGCR
IFI. ZSPACES THEN
SMEXfiIIndaxon .V
036 IF F{Ilndex} WP. Ti-EN
LOOCR
DBGFOR X{.V WMFJ'HEN:

:.F Mmt+ ODOCR F{lh&x) WP.
LOOP:

\ ------------------ “WOODS

:FEFLECT CENTER
F-t FPEGATE \nblaebr-d
senatummon \mn
Exrmoe \ealeulabeM
DEEIFCRIFELECI’M'THENM:

”Poem

mount:

an WW'IM
\mm
xue-(vuove \mpon
MVP-8 mar: (71 nausea-4
"summon \mn
arm \muw

”FOR'W “94 .M;

“VE MF-IR \mw-os

musicians-110)) \vmrnpt
m \muot)
MFCRfHALVMi'Tl-Bd M:

II“ ”((0 Mada-10)) \balpt.
Y( VJKNE \aaveit
M1114» 100 \uywetnr
mono \bywnponent

was >m1notr
mo vmm amass
P-uzP- r+ R121
LOOP
LOOP assume ORDER
Decree-eman- nan .M;
---------------------------- anacnonwoms

“WWW
HMAP >M}mF>W:
>-hau'l 0 M;

and? Min MW:
>-wm1-m;

um Hum-rimlmao

no mwme

POVERFOVE-I F-F2'

F-l-DTF+ Fl FABS mm F>;
.............................. aorEsrwoaos

 

MI-MWW 211

:m (n.~--fl:w~)
Slur-a WM! OIN \flflse
\Ww
\Inenau.m,m
\ml-Ilnn

'01 NM"- < All)

igiig

EFLEC‘TD-hm
F r>-m
F KWFS'YmTl-Gi
HALVE Km
Parmesan-arm
ESE mm
HEM
r>-b‘7
IFSTGEXPESESTGfXW
THE
N1+ISN omen
FEAT core;

\ ----------------------- mamas
:P1 (ea—s7;P)oLpo}Ra.2@mP-2
own ms": H'F+
2) mar-'2 F2'F2'F+
35-FF-F‘1F2/FSWAPOSFF‘F-;
\t1=.5'(x'x+2'y'y+4'z'z-UI"2-0'1

:F‘2 (u—nzrinouPo)
mmUPP-zumm
F2‘F+ SSFF-P’zFZIFSWAPsS-FF‘F-
Mp,ow1}moo}me@P/Psrmrrroos

\12 -[¢r"22'y"2-u)"2-051]'ooa¢'mnulx))‘ 2
\Uuom USE(MYFLNC 101mm

FLOATS

5. swnExuoonm
<1. mum pm
1. smuoznm

5. wmonmr
3. mm mm:
-15$IPLEX{(12))R2!

.1o.suPLE1q(2o}}m21
1. smuzmm
s smauzznm

s. swusxuson m
a. mum }} rear
3. smausznm

212 Ms-mwm WWI]

 

”Harm—“V .4.-.“ .

emvuouum-Mmmemd.

