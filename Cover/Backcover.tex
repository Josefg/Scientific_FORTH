Scientific FORTH: a modern language for scientific computing

FORTH has been called "...one of the best-kept secrets in the computing world." Combining the execution speed of a compiled language with the immediacy and convenience of an interpreted language, FORTH is nevertheless so simple its kernel can be compressed into a few kilobytes of machine code. Many scientists, engineers, and programmers have recognized that FORT H provides the "most direct, revealing and flexible way for controlling computer hardware yet invented," applying FORTH to industrial control, robotics and laboratory instrumentation.

FORTH is the only completely extensible modern computer language. User-defined operators, data structures, commands, functions, and subprograms act precisely like the core operators, data structures and commands -- they are true extensions to FORTH. Moreover, the FORTH compiler is part of the language, available to the user. These features give FORT H enormous abstractive power and elegance of expression. Thus, a FORT H program to solve Unear equations can look as simple as

\begin{lstlisting}
: }}SOLVE ( adr[M] adr[y] --) SETUP TRIANGULARIZE BACKSOLVE ; 
\end{lstlisting}

\textbf{Scientific FORTH} extends the FORTH kernel in the direction ofscientific problem-solving. It is the first book to illustrate advanced FORTH programming techniques with non-trivial applications:

\begin{itemize}
    \item high-speed real and complex floating point arithmetic
    \item numerical mtegration/Monte-Carlo methods
    \item linear equations and matrices
    \item functional representation of data (FFT, polynomials)
    \item function minimization
    \item differential equations
    \item roots of equations
    \item computer algebra
\end{itemize}

\bc{Julian V. Noble} (B.S. Caltech '62; M.A. Princeton '63; Ph.D. Princeton '66) is Professor of Physics at the University of Virginia. His professional interests and publications include nuclear and particle physics, astrophysics, theoretical biology, and numerical methods. H e has been programming digital computers since l961,but became fascinated with personal computers since acquiring his first in 1979. H e now uses FORTH almost exclusively for his scientific work. 