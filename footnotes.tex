% All footnotes are here.

% ****    Preface footnotes ****
\sepfootnotecontent{Pre_01}{Havard Softworks, P.O. Box 69, Springbro, OH45066}

\sepfootnotecontent{Pre_02}{Ecclesiastes, 9.11.
M. Kelly and N. Spies, \textit{FORTH, a Text and Reference} (Prentice-Hall, Englewood Cliffs, N.J., 1986). L. Brodie, \textit{Starting FORTH} (Prentice-Hall, Englewood Cliffs, N.J., 1981).}

% **** Chapter  1 footnotes ****
\sepfootnotecontent{01_01}{This description refers to FORTH's compilation scheme. See, e.g. R.G. Loeliger, \textit{Threaded Interpretive Languages} (Byte Publications, Inc., Peterborough, NH, 1981). We shall have more to say about it in Chatper 2.}

\sepfootnotecontent{01_02}{Although some interpreted FORTRANs such as WATFOR have been developed.}

\sepfootnotecontent{01_03}{The original version of FORTRAN included naming conventions such that names beginning with letters I, J, K, L, M, and N are assumed to be integers, while those beginning with other letters are assumed to be single-precision floating point numbers. Subsequent versions have maintained this convention for backward compatibility.}

\sepfootnotecontent{01_04}{The items between parentheses, (\dots), and following a backslash, "\textbackslash", are comments.}

\sepfootnotecontent{01_05}{A \textbf{stack} is a data structure like a pile of cards, each containing a number. New numbers are added by placing them atop the pile, numbers are also deleted from the top. In essence, a stack is a "last-in, first-out" buffer.}

\sepfootnotecontent{01_06}{L. Brodie, \textit{Thinking Forth} (Prentice-Hall, Inc., Englewood Cliffs, New Jersey, 1984). M. Ham, "Structured Programming", \textit{Dr. Dobb's Journal}, July 1986.}



% **** Chapter  2 footnotes ****
\sepfootnotecontent{02_01}{L. Brodie, \textit{Starting FORTH}, 2nd ed. (Prentice-Hall, NJ, 1986), referred to hereafter as \SF.}

\sepfootnotecontent{02_02}{L. Brodie, \textit{Thinking FORTH} (Prentice-Hall, NJ 1984), referred to hereafter as \TF.}

\sepfootnotecontent{02_03}{M. Kelly and N. Spies, FORTH: a Tea and Reference (Prentice-Hall, NJ , 1986), referred to hereafter as \FTR.}

\sepfootnotecontent{02_04}{Successive words in the input stream are separated from each other by blank spaces, ASCII 20hex, the standard FORTH delimiter.}

\sepfootnotecontent{02_05}{We will explain about the stack in 2.3.}

\sepfootnotecontent{02_06}{Since FORTH uses words, when we enter an input line we say the corresponding phrase.}

\sepfootnotecontent{02_07}{This level could be either the outer interpreter or a word that invokes \bc{NEW-WORD}.}

\sepfootnotecontent{02_08}{defined as \bc{: -ROT ROT ROT ;}}



% **** Chapter  3 footnotes ****



% **** Chapter  4 footnotes ****



% **** Chapter  5 footnotes ****



% **** Chapter  6 footnotes ****



% **** Chapter  7 footnotes ****



% **** Chapter  8 footnotes ****
\sepfootnotecontent{08_01}{If a rectangle lies below the horizontal axis, its area is considered to be \textbf{negative}.}

\sepfootnotecontent{08_02}{f'(x)is the slope of the line tangent to the curve at the point $x$. It is called the first derivative of $f(x)$.}

\sepfootnotecontent{08_03}{This is not strictly correct: one could use a differential equation solver of the "predictor/corrector" variety, with variable step-size, to integrate Eq. 4. See, e.g., Press, et al., Numerical Recipes (Cambridge University Press, Cambridge, 1986), pp. 102 ff.}

\sepfootnotecontent{08_04}{That is, this statement defines $\bar{f}$.}

\sepfootnotecontent{08_05}{The word % pushes what follows in the unput stream onto the 87stack, assuming it can be interpreted as a floating point number.}

\sepfootnotecontent{08_06}{J.M. Hammersley and D.C. Hanscomb, \textit{Monte Carlo Methods} (Methuen, London, 1964).}

\sepfootnotecontent{08_07}{See, \eg, R. Sedgewick, \textit{Algorithms} (Addison-Wesley Publishing Company, Reading, MA, 1983), p. 85.}

\sepfootnotecontent{08_08}{An ancient Greek mathematician known for one or two other things!}

\sepfootnotecontent{08_09}{See, \eg Sedgewick, \textit{op. cit.}, p. 11.}

\sepfootnotecontent{08_10}{Most FORTHs do not permit a word to call itself by name; the reason is that when the compiler tries to compile the self-reference, the definition has not yet been completed and so cannot be looked up in the dictionary. Instead, we use \textbf){RECURSE} to stand for the name of the self-calling word. See Note 14 below.}

\sepfootnotecontent{08_11}{TRACE is specific to HS/FORTH, but most dialects will support a similar operation. \textbf{SSTRACE} is a modification that sinde-steps through a program.}

\sepfootnotecontent{08_12}{For examlpe, it is often claimed that removing recursion almost always produces a faster algorithm. See, \eg Sedgewick, \textit{op. cit.}, p. 12.}

\sepfootnotecontent{08_13}{For generality we do not specify the integration rule for sub-intervals, but factor it into its own word. If we want to change the rule, we then need redefine but one component (actually two, since the Richardson extrapolation --see Appendix 8.C-- needs to be changed also).}

\sepfootnotecontent{08_14}{We may define \bc{RECURSE} (in reverse order) as
\begin{lstlisting}
    : RECURSE ?COMP LAST-CFA , ;
    : ?COMP STATE @ 0= ABORT" Compile only!" ;
    : LAST-CFA LATEST PFA CFA ; IMMEDIATE
    \ These defintions are appropriate for HS/FORTH
\end{lstlisting}
Note that \bc{RECURSE} is called \bc{MYSELF} in some dialects.}

\sepfootnotecontent{08_15}{We use the generalized arrays of Ch. 5 \S\ 3.4; A, B, and E are fp \#'s on the fstack, TYPE is the data-type of f(x) and INTEGRAL.}

\sepfootnotecontent{08_16}{The memory usage is shout the same: the recursive method pushes limits, \etc onto the fstack.}

\sepfootnotecontent{08_17}{"Analytic" meants the ordinary derivatiove $df(z/dz$ exist. Coonsult any good text on the theory of functions of a complex variable.}

\sepfootnotecontent{08_18}{This comes from the Chebyshev polynomial representation for $sin(x)$. See, \eg, Abramowitz and Stegun, \textit{HMF}, \S\ 4.3.104.}

\sepfootnotecontent{08_19}{Although the 80x87 already uses a compact represtation of the trigonometric functions and is thus fairly hard to beat, especially if high accuracy is demanded.}

\sepfootnotecontent{08_20}{to prevent \textbf{aliasing}.}

\sepfootnotecontent{08_21}{See, \eg, DE. Knuth, \textit{The Art of Computer Programming}, v.2 (Addison-Wesley Publishing Co., Reading, MA, 1981) p. 642.}

\sepfootnotecontent{08_22}{This example is taken from the article “FORTH and the Fast Fourier Transform" by Joe Barnhart, \textit{Dr. Dabb's Journal}, September 1984, p. 34.}

\sepfootnotecontent{08_23}{}

\sepfootnotecontent{08_24}{}

\sepfootnotecontent{08_25}{}

\sepfootnotecontent{08_26}{}

\sepfootnotecontent{08_27}{}

\sepfootnotecontent{08_28}{}

\sepfootnotecontent{08_29}{}



% **** Chapter  9 footnotes ****
\sepfootnotecontent{09_23}{}

\sepfootnotecontent{09_24}{}

\sepfootnotecontent{09_25}{}

\sepfootnotecontent{09_26}{}

\sepfootnotecontent{09_27}{}

\sepfootnotecontent{09_28}{}

\sepfootnotecontent{09_29}{}



% **** Chapter 10 footnotes ****
\sepfootnotecontent{10_23}{}

\sepfootnotecontent{10_24}{}

\sepfootnotecontent{10_25}{}

\sepfootnotecontent{10_26}{}

\sepfootnotecontent{10_27}{}

\sepfootnotecontent{10_28}{}

\sepfootnotecontent{10_29}{}



% **** Chapter 11 footnotes ****
\sepfootnotecontent{11_23}{}

\sepfootnotecontent{11_24}{}

\sepfootnotecontent{11_25}{}

\sepfootnotecontent{11_26}{}

\sepfootnotecontent{11_27}{}

\sepfootnotecontent{11_28}{}

\sepfootnotecontent{11_29}{}
